\input{ main-style }

\title{Assignment P2\\}

\authoremail{Jeremy Martinez}{jmartinez91@gatech.edu}

\begin{document}
\maketitle
\thispagestyle{fancy}

\section{Question 1}

\subsection{Task 1 - opening my text editor}

\paragraph{Definition}

The task is opening my text editor (Sublime Text 3) to the specific directory I want to edit. The goal of task 1 is to have the correct directory opened in my editor. The interface for task 1 is within the context of my operating system, which is Mac OS. The interface is my keyboard, keypad, and screen. The object of task 1 is my laptop (MacBook Pro).

\paragraph{Directness and invisibility of interaction}
This task begins with an open laptop already running my OS. In this context, I either have a blank desktop, or an open (perhaps multiple) windows with running applications. I am interacting with the object directly, manipulating its interface via the keyboard/keypad. The distance from the object to the task is all via the OS, which has keyboard shortcuts, and a more traditional medium of finding an application through a file explorer/finder. Since I do this task nearly every day, the interface is nearly invisible to me. This is because I use keyboard shortcuts (CMD + Space, search Sublime Text 3). Focusing on this task only becomes invisible through repetition, and even getting to the final state of opening the directory you want still has friction. The application, by default, opens the last directory you had open. This can make the interface feel invisible. However, if you need another directory, then you must find it by clicking through the entire directory tree (not ideal). The number of actions necessary for this task are minimal, so there wasn't a time where I thought about the subtasks in this interface significantly more.

\subsection{Task 2 - buckling my seatbelt}

\paragraph{Definition}
The task is (to travel safe, via) buckling my seat belt in my car specifically, however, I will compare this to buckling my seat belt in other settings, too. The goal of task 2 is to have your seat belt buckled before you are in a position to be harmed by an accident. The interface for task 2 is the layout of

\clearpage

your seat, as well as surrounding indicators, such as a dashboard beeping reminder. The object of task 2 is your car (it's not driving, so I'm still obeying the rules of the assignment!).

\paragraph{Directness and invisibility of interaction}
The task (fastening seat belt) is physically located within the object (car)? You are directly manipulating the object, since the task is attached to the object. In terms of invisibility, I spend little to no time thinking about this task. So little in fact, that sometimes I forget, and the indicator on the dashboard beeps/alarms when I forget. This is an intuitive reminder that I can immediately comprehend since the icon representing the task directly depicts the (piece) object the object, the person/seatbelt. The interface became invisible; however, this could be because of the amount of repetition involved with this task. The red beeping/alarming icon is so engrained into our action associated with driving that it has become second nature. This is contextual, since I recently flew in a plane and helicopter, both of which, I had to be reminded at some point by another passenger/flight attendant to fasten my seat belt (no beeping reminder). There was a time when I thought more about this interface, and that was prior to this beeping/alarm being universally consistent in all newer vehicles.

\subsection{Task 3 - getting an espresso}

\paragraph{Definition}
The task is to select the espresso option and pour the espresso. The goal of task 3 is to have an espresso. The interface for task 3 is the computer screen of the coffee machine. The object of task 3 is the espresso maker, which is comprised of two containers filled with coffee beans, a computer screen, a milk deposit, and an internal component that does all of the grinding/steaming/etc.

\paragraph{Directness and invisibility of interaction}
The interactions are made directly to the screen of the object. I am manipulating the object at a short distance (figuratively, since the screen is attached to the object) via the interface since the internals of how the espresso is made are all abstracted away from me. In terms of invisibility, I spent a couple subtasks thinking about the interface while performing the task: pressing the screen to navigate away from the lock screen, then slide (like an iPhone home screen) across pages of the desktop to find the espresso button, then clicking button and placing my cup underneath the spout. When I focus mostly on the task, the interface becomes invisible and the process of stepping through the subtasks feels intuitive. When I first used this interface, I did have to think about the interface more. I would slide across all pages, looking for what's available and evaluating the options. With time, this interface starts to flow more naturally.

\subsection{Task 4 - moving my desk up/down}

\paragraph{Definition}
The task is moving my electronic stand up desk from sitting to standing (and vice versa). The goal of task 4 is to either sit or stand at my desk. The interface for task 4 is a panel to the front right of the desk with a handful of buttons. The object of task 4 is my desk.

\paragraph{Directness and invisibility of interaction}
My interactions are directly on the object. I am 100 percent manipulating the object directly. To do this, I would physically pull/push the desk top up/down. However, by clicking the buttons to go up and down directly on the panel attached to the desktop, I am interfacing directly with the object to an extent. In terms of invisibility, I spend absolutely no time thinking about this interface and have the impression I am interacting directly with the object. When I focus mostly on the task, the interface becomes invisible through good design. The was never a time I had to think about this task (zero learning curve).

\subsection{Task 5 - saving a song on my Spotify}

\paragraph{Definition}
The task is to save a song on Spotify. This task is analyzed in the context of already having the Spotify app open and a list of songs present on the screen. The goal of task 5 is have a song saved to your library (meaning it is available offline). The interface for task 5 is the Spotify app. The object of task 5 is the smartphone (iPhone).

\paragraph{Directness and invisibility of interaction}
The interaction is directly on the screen of the object. I am manipulating the components within Spotify which are directly surfaced on the objects screen. In terms of invisibility, I spent a decent amount of time trying to find the actions within the interface. I have to click the three dots to the right of the row of the song I want to save. If I focus mostly on the task, the interface can start to feel less visible the more that I learn from it. Just recently I learned a shortcut to this action (swiping the row to the right). However, up until recently this was a task involving a not so intuitive list of subtasks (an experienced user would recognize the three dots as an indication to more available actions).

\section{Question 2}

The task I do on a regular basis is using my propane fueled BBQ to cook food. The goal of this task is to cook food. The components to this object comprise the propane tank, the grill, the BBQ's frame and cover, the propane knobs (front, middle, back) and igniter. The components I used to focus on was the sequence of engaging with each of these components. Sometimes I would toggle back-and-forth between the igniter, knobs, and propane tank when the grill would not light.

Now, after having interfaced with this specific grill a handful of times, I have memorized and learned the internals to how this system functions. The grill must be supplied with propane to function. This is the base-line requirement, meaning the propane tank must be opened first. Then, the propane must be flowing through the valves, which is what the knobs mediate between the user and underneath the grill. Once the fuel source is being fed through the pipes under the grill, a spark must be made for a fire, so I spark the igniter. Having an in-depth knowledge allows you to grasp the intuition behind many grill's interfaces of the same construct.

\paragraph{5 tips: invisible interfaces}
\begin{enumerate}
\item
  Use affordances - Places where the visual design of the interface is how it is supposed to be used - \textbf{the grill does a good job of this.}
\item
  Know your user - Invisibility could mean different things - \textbf{new users may not have an in-depth knowledge of the inner-workings}
\item
  Differentiate your user - Provide multiple ways for accomplishing tasks to differentiate between users - \textbf{(N/A)}
\item
  Let your interface teach - Other than tutorials or manuals. Have your interface teach aspects of itself - \textbf{the interface teaches the user how the components work together}
\item
  Talk to your user - Ask them what their thinking while using an interface - \textbf{N/A}
\end{enumerate}

This system could be redesigned to place the knobs in a fashion the better reflect the fact they manipulate different sections of the grill. Placing the knobs on the grill hood could place them directly in front of the user's face. Also, combining the fuel release (knob) in with the igniter (detecting whether there is a flame or not) and triggering the igniter if there is no flame will reduce the gulf of execution. Placing a light/signal on the grill hood indicating when the grill is lit will reduce the gulf of evaluation as well.

\section{Question 3}

\subsection{Task - using an advanced treadmill}
including heart monitoring, calorie-counting, rate variation, etc. will be our selected task.

\begin{itemize}
\item
  Visual - the user is visually engaged by the system through the computer screen mounted on the front of the treadmill. At my gym, this system looks a lot like the Windows 10 OS (tile-based app presentation) where the user can select between TV, YouTube, workout simulations, etc.
\item
  Auditory - the user is NOT engaged with audio unless they plug in headphones to the system (which I have never done). A user might do this to hear the audio of the video (or whatever app) they are watching on the screen. In this fashion, it is solely used to complement the visual.
\item
  Haptic - the user is engaged haptically, in my experience, only through a fan built in on the front of the system. The belt move is another indication that the exercise has begun. However, it seems this exercise to discuss complementary features to the treadmill, whereas running (on a belt) is the main function of the treadmill.
\end{itemize}

\paragraph{Visually}
the system engages the user the most (compared to the other senses). The system has a sole purpose of exercise, and so introducing a notification system in the UI based around this would be a subtle approach to this. When a user hits a mile-marker, surface a push notification in the corner, which could also give a small endorphin boost to the user.

\paragraph{Auditorily}
any addition to the system would improve the user experience (I am not considering the experience with headphones since this requires extra equipment). This could be tricky, since in a common area like a gym, you don't want to contribute to noise pollution. Putting sensors for when you are close to the edges (front/back) of the belt and a small alarm to let the user know they are about to fall off may be a useful safety measure. Another, more practical solution tangential to the headphone issue, is to support Bluetooth headphones. I use wireless headphones, so connecting to the audio at the treadmill at my gym is not an option (Bluetooth is not supported). Also, increase in running pace could be accompanied with a click (which I've seen in some, but not all systems).

\paragraph{Haptically}
there is a lot of room for improvement here, like with auditory. The heart sensor should vibrate when your rate is available. You have to grab the sensor and hold on for N number of seconds. However, what exactly the value of N is unknown, and so we keep glancing down at the screen for a visual indication that our heart rate is available. A haptic interface here would improve the users awareness of the feature.

\paragraph{Finally}
imagine if a treadmill could engage you through the sense of smell. Picture the following:

\begin{enumerate}
\item
  a visual representation on the screen of you running through a trail somewhere in the amazon (recorded with a GoPro) (visual),
\item
  you are connected via wireless headphones and can hear natural sounds of the rainforest (auditory),
\item
  the incline changes to whether the runner in the video is going uphill/downhill (haptic),
\item
  you have a fan on the front of the system blowing air (haptic), and
\item
  from near the fan, smells of the rainforest fitting the setting you are running through are spritzed (smells)
\end{enumerate}

This system could engage you through four senses to recreate trail running. The biggest issue with treadmills is that it does not engage their user. This is why many people describe running on a treadmill as monotonous. With a system like this, one could create an entertainment product focused around running on a treadmill.

\section{Question 4}

\paragraph{Reducing cognitive load in interface design}
\begin{itemize}
\item
  Using multiple modalities - engage multiple systems to prevent the user from becoming bored or their attention to wonder
\item
  Letting the modalities complement each other - presenting two different things in two different forms can increase cognitive load, making it more difficult for the user to grasp the subject matter. Rather, each modality should complement and explain the same concept.
\item
  Giving the user control of the pace - this will give autonomy and reduce stress on the user
\item
  Emphasizing essential content while minimizing clutter
\item
  Offloading tasks from the user onto the interface - keep track of required actions from the user, and attempt to trigger actions automatically if necessary, information to do so was provided earlier in the interface
\end{itemize}

\subsection{Using multiple modalities & Letting the modalities complement each other}
The most obvious situation this occurs in my professional life is at work with PowerPoint presentations. Often times, a presenter will place text on a screen, and then regurgitate the words on the screen in their speech. Even though this is a visual representation paired with an auditory, it feels a bit like cheating since the visual is a transcript of the auditory. They are also not really complementing each other, since it does not offer a new medium for the user to digest the same concept. The presenter should use their interface to capture the user's attention and have them process their subject matter in a different way. The slide show should offer animations/diagrams that illustrate what they are saying so the user can see their concept in practice while they are defining it.

\subsection{Giving the user control of the pace}
I am studying Spanish in my free time. I user the Rosetta Stone mobile app to do this most nights before I go to bed. This mobile app is excellent, and I highly recommend it. However, it does one thing automatically that the user does not have control over.

It paginates each task with four sub-tasks. For example, when doing a task based on pronunciation, it will give you four sentences to pronounce. After the fourth, it pauses, then continues to the next. There is a pause button to the side you can press in between one of these four sub-tasks. However, after you complete the fourth, all of the text appears above the image and it would be useful to be able to pause here to review visually, because once you see the vocabulary it is easier to comprehend/retain the sentence. However, this is not supported, and so the user does not have control over this sequence. A fix to this is would be to provide support to pause the screen in that brief period when the text appears, but the module does not continue (see figure for context).

\begin{figure}[H]
  \centering
  \includegraphics[scale=0.25]{figs/before}
  \caption{Rosetta stone module before final subtask is completed.}
  \label{fig::1}
\end{figure}

\begin{figure}[H]
  \centering
  \includegraphics[scale=0.25]{figs/after}
  \caption{Rosetta stone module after final subtask is completed.}
  \label{fig::2}
\end{figure}

\end{document}

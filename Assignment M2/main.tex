\input{ main-style }

\title{Assignment M1\\}

\authoremail{Jeremy Martinez}{jmartinez91@gatech.edu}

\begin{document}
\maketitle
\thispagestyle{fancy}

% You may include a new tex file here using:
% \input{<file_name>.tex}

\begin{abstract}
I will be focusing on a website’s navigation bar. The nav bar is usually the main call to action when visiting a homepage of a web application and provides context for what actions are available. I will compare navigability between desktop/mobile versions of a website as well.
\end{abstract}


\section{Needfinding - Think Aloud}

\paragraph{Summarizing the results}
The interfaces studied in this needfinding exercise are listed below:

\begin{enumerate}
\item
  \textbf{Spoiled Spa and Salon} - desktop browser - a neighborhood hair salon and spa
\item
  \textbf{Firefly Atelier Hair Salon} - mobile browser - a neighborhood hair salon
\item
  \textbf{Twitch.tv} - desktop browser - an online video game streaming platform
\item
  \textbf{Groupon} - mobile app - an online coupon service (mobile app studied instead of mobile browser)
\item
  \textbf{YouTube} - mobile app - an online video hosting service (mobile app studied instead of mobile browser)
\end{enumerate}

Most users found the use of symbols very effective and were able to identify tasks based on these. Users commented on when and where screen real estate was used effectively, or they felt that it was wasted. The Firefly Atelier Hair Salon was an example of this in that they used too much text, had no clear CTA's on their homepage (and even had a dismissable notification that made the app difficult to use). Spoiled Spa and Salon was the opposite of this in that their home page surfaces buttons to steer you in a clear path immediately. Both salons mobile browser applications leveraged the \link{https://fontawesome.com/icons/bars}{hamburger} in the upper left, which was recognizable to everyone given the web standard.

\clearpage

The search bar seemed to always dominate the attention of the user in the nav. Though this functionality was used whenever it was available, it did dwarf other aspects of the nav bar. Every user found the doulbe nav bar from YouTube very useful and easy to use given their thumb placement. This is interesting because using two navs ensure your user always has direct access to the action they want to accomplish (given this only has a limit number of tasks).

When using so much screen real estate for the nav, one must be efficient in using the main section of their application. Leveraging well known icons, and using visuals over text to communicate whenever possible it key. All users seemed to be okay with seeing text on the "who we are" section of Firefly's website, since they had clicked to a section where they would expect a lot of text.

The script of some of the focal points of the think aloud exercises can be found in Appendix B. The raw notes of the think aloud exercises can be found in Appendix C.

\paragraph{Summarizing the takeaways}
\begin{itemize}
  \item
    \textbf{Avoid text}. It is vital to keep the text to a minimum whenever possible, use images/icons to convey meaning. Stay consistent with these representations
  \item
    \textbf{Top & bottom nav}. Surface a nav on both the top and the bottom (fixed on the screen).
  \item
    \textbf{Dismissable items}. Avoid dismissable pop-ups whenever possible.
  \item
    \textbf{Search is first}. Search is good, but overpowering and demanding of the users attention. Include this, but de-emphasized in some way.
  \item
    \textbf{Landing page}. Show calls to actions immediately. The nav may not be enough, the user came here for a reason, try to guess what that is and put it in front of their face quickly!
  \item
    \textbf{Categorize}. Categorize everything in an intuitive way, and stay consistent with this. The user will subconciously pick up on this structure and teach themselves to navigate the app based off of it.
  \item
    \textbf{Emphasize more common actions}. If your nav requires too much, do what Twitch does. Surface the most common actions to appease the many, offer a "more" option with a subnav for the more technical.
\end{itemize}

\paragraph{Avoiding bias}
In order to avoid \textbf{recall bias}, the user was asked to reflect on actions as often as possible, rather than pre-emptively asked to explain why they're performing an action. Doing the latter can sometimes change their metacognitive process to give them more awareness than they normally would. Also, we used a wide array of application types in order to prevent tunnel vision on one type of feedback. Careful concern was given to avoid assisting the user in any way to avoid \textbf{observer bias}.

To avoid potential \textbf{confirmation bias} of only analyzing a small subset of hair salon web page layouts, I will conduct a think aloud exercise on two extra interfaces (from a different type of industry) for desktop application and two different mobile apps (not mobile browser) to compare/contrast how these different platforms compare.

Finally, since these user interfaces were not developed by myself, the user had no incentive to compliment them, thus avoiding \textbf{social desireability bias}.


\section{Existing user interfaces}

\paragraph{Summarizing the results}
For this needfinding exercise, I will focus on two hair salons in the Vancouver, WA area (which is the location of the hair salon this study is for) as well as web app interface from an entirely separate industry. The interface of concern is their business website. The reason for analyzing these two subsets is to see what the local business do well to establish an intimate relationship with their neighborhood, while also recognizing what enables the chain salons to work at scale. I will discover all of these salons profiles via Google search, and more specifically, the website button next to their Google maps profile. When comparing these two interfaces, I am looking for the following attributes
\begin{itemize}
\item
  \textbf{Content}. What content are they serving? How is this content relevant to their intended audience? This will be good to cross reference with results from our survey
\item
  \textbf{Landing page}. What do they deem most important to show their customers immediately? Is this useful? Are there obvious calls to action to find more information?
\item
  \textbf{Calls to action}. How are these (if any) calls to action displayed? How easy is it to use on a desktop browser? Mobile? Tablet?
\item
  \textbf{Content per tab}. Does the nav bar do a good job of displaying enough, but not too much information? If it's too much, how could it be easier to navigate?
\end{itemize}

\subsection{Spoiled Spa and Salon}

\paragraph{Content & landing page}
for this salon serves a nav at the upper-right part of the screen for desktop and a \link{https://fontawesome.com/icons/bars}{hamburger} at the upper-left for tablet/mobile. The tabs include services, gift certificates, products, specials, policies, and appointment bookings. The content of the page below contains contact information, calls to action (buttons) on their most common services, and a upsell for gift card certificates..

\paragraph{Calls to action}
for the services provided all have consistent layouts. They list out their services, tiers of each services, and prices associated with these services.

\paragraph{Content per tab}
The services tab from the nav bar has a sub nav matching the services on the landing page. The gift certificates tab has an grid of tiles all upselling gift certificates. They use a service calls Vagaro to fulfill these. The products tab allows you to add hair products to a cart (also backed by Vagaro). Specials tab lists off coupons for the various services they provide. Policies serves like an FAQ to cancellation policies, age restrictions, etc. Book an appointment has a easy to use interface listing out services and available times for appointments (also backed by Vagaro).

\subsection{Firefly Atelier Hair Salon}

\paragraph{Content & landing page}
for this salon serves a nav at the upper-right part of the screen for desktop and a \link{https://fontawesome.com/icons/bars}{hamburger} at the upper-left for tablet/mobile. The tabs include home, our team, products, services, contact, promotional and events. The content of the page below contains a video on loop showing the salon on an average day giving insight to their day-to-day operations. As you scroll down, the landing page (home page) also has three calls to action (in line with the nav tabs) for services, meet our team, and contact.

\paragraph{Calls to action}
on the video of the salon there is a downward arrow indicating to check out the content below. They have a short description. The three calls to action map to the three tab in the nav bar.

\paragraph{Content per tab}
The products tab here simply lays out what products they offer/use (pictures only). There is no call to action on this page. The services tab lists out what they offer in terms of services. Each service has it's own page listing out the prices for that particular service. The contact form lists out contact information, GPS maps of their location, and an email form. "Who we are" contains a YouTube video introducing the team that works there

\subsection{Twitch.tv}

\paragraph{Content & landing page}
for this web application contains a nav at the top bar, which remains the same in tablet and mobile, however, to conserve space, they reduce the tab to icons. The nav includes three tabs in the upper left (most popular three) discover, browse, and type prime, accompanied with a ... symbol to indicate all other actions. Then it includes a search bar in the center. Finally, it includes a login/signup tab in the upper right (we will ignore this for now since login/signup is not relevant to our use case).

\paragraph{Calls to action}
on the home page are to select different video game streams that are currently happening. This is to appease to their most common user. As you scroll, you just find more and more types of streams.

\paragraph{Content per tab}
The browse button takes you back to the same page (essentially a home page button). The browse button takes you a page focused around categorizing streams and enabling the user to easily sift through their wide array of content. Twitch prime is an upsell for their paid-for service.

\subsection{Summarizing the takeaways}
The salon interfaces are useful becasue they give us a persepctive of what this user type will expect to see. They will be use to seeing content similar to their competitors. However, they are narrow minding. Twitch offers an interesting perspective on the implementation. They have offered a few easy-to-find tabs in the nav for 99% of their users. Then for everyone else, they've provided an "advanced" option where they can find less popular options. This is a useful strategy that we can leverage to appease the many, while still making functionality available to the few.

\paragraph{Avoiding bias}
Tunnel vision can occur during this needfinding exercise because one can focus too much on replicating their competitors. While I believe the competitors approach contains useful parts, we will try to only retain the content they contain, and approach the navigation aspect from an entirely new perspective. The goal is to get the user to feel the most comfortable in ther interface. We are going to pay extra attention to not restrict ourselves to this style of navigation implementation if a better one exists.


\section{Needfinding - Survey}

\paragraph{Summarizing the results}
The survey results were very insightful. The analysis of all of the survey responses can be found below in Appendix E (actual survey in Appendix D). This survey is particular useful to tell use what our users want to see when they visit the site. Most of the users wanted to see content like portfolio, types of products/services, bio of stylists, prices, online booking, free consultation, blogs, hours, contact info, business history, etc.

The users, whether new or existing, wanted to see relatively the same content. However, it is interesting to see how this subtly changes. It could introduce a different kind of flow to the navigation. Imagine, asking a user if they are new or returning, then rendering content based on that answer. While I do not think that implementation is right for this project, it is something worth exploring for other projects in the future.

\paragraph{Summarizing the takeaways}
\begin{itemize}
\item
  \textbf{What users want to see}. portfolio, types of products/services, bio of stylists, prices, online booking, free consultation, blogs, hours, contact info, business history, etc.
\item
  \textbf{Where they are}. They're on their phones.
\item
  \textbf{Pop ups are disruptive}. Avoid these at all costs. They feel like ads.
\item
  \textbf{How do they decide?} Reviews, location, and website aesthetics play a large role.
\item
  \textbf{Repeat customers objectives}. They have one intention, so serve the CTA quickly.
\item
  \textbf{What do they really want?} Users may say they want to see a wide array of things, but that's all background noise (and possibly not even true). They want to know prices, services, products, location, contact, book an appointment. This is their \textemp{actual} motive for arriving at your website.
\end{itemize}

\paragraph{Avoiding bias}
Survey questions were wording an a way to avoid \textbf{observer bias}. This survey was distributed in a casual fashion to friends, family, and classmates so as to avoid \textbf{voluntary response bias}. In light of this manner, the questions were extremely generic to avoid \textbf{social desirability bias}.


\section{Data inventory}

\paragraph{Our expectation on user types}
The intended user for this application is adults in general. However, the current customer base will be the heavy focus of whom this experience will be designed for, which includes men and women ages 30-60. We anticipate the majority of the target audience to be older than millennials.

When evaluating the intended audience, the following is taken into account

\begin{itemize}
\item
  \textbf{Demographic of our audience} fits (for the most part) men and women aged 25-65. We are choosing this because, for the specific salon owner I am working with, their client-base has matched this for the last 20 years.
\item
  \textbf{Their level of expertise} will be assumed to be not very technical. This audience will be in the grey area between a millennial and geriatric. We will assume for this reason that the user does not have a strong grasp on the differences between types of browsers, versions of browsers, or experiences between mobile vs desktop browsers. The UI should be styled in such a way that the look and feel is familiar, yet clearly geared to be optimized on the platform at hand.
\item
  \textbf{Their motivation for engaging in the task} will not be singular. The number of reasons one might visit a hair salons website could be for general salon information, hours of operation, salon team members, products/services they offer, schedule an appointment, contact information. We can try to anticipate our users needs; however, we must assume they have any one of these intentions. Therefore, having an intuitive navigation inside the application is crucial to naturally enabling the user to accomplish the task they came to do without any friction.
\end{itemize}

\paragraph{Who are the users?}
Our users are new and existing customers. Depending on which, they may have slightly different goals, however, this is not a drastic difference. Our users are men and women between the ages of 25-65.

\paragraph{Where are the users?}
They are on their phones! A small percentage will be on their laptops/desktops, however, this is a clear minority.

\paragraph{What is the context of the task?}
The context of the task is landing on a business home webpage (usually from Google/Yelp/some other search engine like service).

\paragraph{What are their goals?}
They typically have one goal - to find a new hair salon or to get information on their existing hair salon.

\paragraph{What do they need?}
They need a responsive app (for both mobile/tablet/desktop). They need clear calls to action on their landing page to quckly accomplish their task. They need more icons/images and less font. They need a nav bar on both top and bottom so these CTA are always readily available.

\paragraph{What are their tasks?}
Their tasks are to find a salon portfolio, types of products/services, bio of stylists, prices, online booking, free consultation, blogs, hours, contact info, business history, etc.

\paragraph{What are their subtasks?}
Subtasks for these embody the steps in the gulf of execution. With the right design and nav bar, these subtasks can be reduced everytime to one or two clicks.


\section{Defining Requirements}
This interface is for a local hair salon in Vancouver, WA. It must provide functionality to address all of the desired tasks laid on in the data inventory. It should accomplish this via a well designed nav bar, must be responsive for all platforms browsers. With the right design, the only learnability goals we'll have should be icon-to-task association.

To evaluate the success of a prototype, we may consider adding a survey into the website and ask the user if their needs were met. We can also add frontend metrics on page visits, link clicks, etc. We will want, for example, a large percentage of page visits to the contact page to result in a telephone href link being clicked. Similar metrics would be gathered for appointment bookings


\section{Continued Needfinding}
The next iteration of needfinding will be more think-aloud exercises with the prototype (user testing). This will let us know if our design is actually accomplishing what we intend. A/B testing on different implementations (\textbf{Analysis of existing data logs}). Using a tool (\link{https://www.fullstory.com/}{Fullstory}) to watch live user sessions (\textbf{naturalistic observation}). Interviewing people on the product as well as at the salon for which this website is for to ask their clients directly. This will help us customize and help refine our specific user base.


\section{Appendix A: data protection & privacy}

\begin{itemize}
\item
  \textbf{explain the goal} of the exercise
\item
  \textbf{Explain what we'll be doing} and the sub-tasks involved
\item
  \textbf{Ask the user} if they are okay with me recording the exercise and state that it will only be used for my reflection as a means of extensive note taking
\item
  \textbf{Explain how the data} will be stored and the time to live (ttl).
\item
  \textbf{All of their information will be anonmized} and there will be nothing to correlate any data back to a specific participant. Their email/contact information will not be shared.
\item
  \textbf{Ensure they are aware} the study can be halted at any time if they so choose
\item
  \textbf{Offer to share the final deliverable} with them once the study and product is complete
\end{itemize}


\section{Appendix B: Think Aloud Script}
Prior to this script, the user is sat down (with me) with the interface in front of them (mobile/desktop - omitting tablet for now). The user is instructed to articulate their though process every inch of the way in attempting to accomplish their task. Tasks between mobile/desktop will be the same.

\begin{itemize}
\item
  What is your initial impression of this [welcome] page? Too much text, not enough text? Is the overall design inviting?
\item
  What is your main objective upon landing on this [welcome] page?
\item
  Is there a clear call to action? Is it clear how to achieve your initial objective?
\item
  How clear is it where the navigation tool to the website is?
\end{itemize}


\section{Appendix C: Think Aloud Notes}
 Both participants were asked to use all five of these applications each with different tasks in mind. In an attempt to have a controlled experiment, all of these tasks are relatively similar - use the navigation bar to complete some task.

\subsection{Participant A}

\begin{enumerate}
\item
  \textbf{Spoiled Spa and Salon} - desktop browser - new client attempting to discover more information about this hair salon
  \begin{enumerate}
  \item
    Likes the fact that everything is available, book now available.
  \item
    The immediate calls to action were good. Surfaces immediately the ability to accomplish what you came here to do.
  \item
    Missing the "who we are" section. This is necessary to humanize the business.
  \item
    Objectives seemed to revolve around pricing, booking, location information.
  \item
    Nav bar was obvious and easy to find. Using industry standard symbol made finding it intuitive.
  \end{enumerate}
\item
  \textbf{Firefly Atelier Hair Salon} - mobile browser - returning client attempting to find contact info/store hours/stylist information
  \begin{enumerate}
  \item
    Bare - poor design with screen real estate. Notficiations that prompt user to book appointment make site nearly unusable. Forces you to dismiss
  \item
    Scrolling a lot to find the content the user wants. Too much text.
  \item
    No clear CTA on screen until you scroll. User assumes they must find the nav and the CTA is there (when actually it is also on the home page if you scroll)
  \item
    Hours of operation is located at bottom in footer. Would make more sense to be on page with location (or possibly both).
  \item
    Nav bar is consistent with industry standard. \link{https://fontawesome.com/icons/bars}{hamburger} in upper left corner.
  \end{enumerate}
\item
  \textbf{Twitch.tv} - desktop browser - attempting to find a stream from a specific user
  \begin{enumerate}
  \item
    First impression - main CTA is the search bar in the top-middle of the app. Likes suggestions/auto complete, helps with odd spelled searches
  \item
    A lot going on in UI, lots of options, categorized really well.
  \item
    Categorization is consistent and everywhere, making sifting through the content simple. Content is icon/image/visual based evverywhere, not much text.
  \item
    In looking for stream from user, clear CTA with search bar.
  \item
    On user page, the CTA to accomplish their goal is obvious. The UI satisfies the most common tasks first/foremost. More technical functionality still exists, just emphasized less.
  \item
    Nav bar is in upper left, clear and obvious how to get what the user wants out of it. Easy to recognize the "..." symbol.
  \end{enumerate}
\item
  \textbf{Groupon} - mobile application - find a coupon for a movie theater
  \begin{enumerate}
  \item
    Layout out well - immediately drawn to the search bar as the main CTA. Assistance for search bar is nice.
  \item
    Screen real estate is not used well. The content being emphasized is not relevant to user.
  \item
    Objectives all revolved around finding coupons for business. Using main CTA (search bar), the search bar had several good features - location. Search results are too broad and confusing
  \item
    On product page, CTA is very clear and fixed on the screen (purchase). User finds this useful that it is always available.
  \item
    Nav at times can be difficult to find. On some pages, nav disappears entirely in lieu of a back button "<" in upper left corner.
  \end{enumerate}
\item
  \textbf{YouTube} - mobile application - Send feedback on an issue you had with the product
  \begin{enumerate}
  \item
    First impression, the UI is centered around videos - solving the most common objective of the product.
  \item
    CTA to accomlish task is not too clear. User had to click around to find a button in the nav that satisfied this (user profile picture in upper right).
  \item
    User is okay with this gap in knowledge of the UI since their task is uncommon.
  \item
    Nav is partially clear, very available. The placement is convenient to how the user holds their phone.
  \item
    Likes split nav, ensure his options are always available. Especially on the home page.
  \end{enumerate}
\end{enumerate}


\section{Appendix D: Survey}
\begin{enumerate}
\item
  What are you looking for in a hair salon?
\item
  What helps one salon stand out from another? What improves your experience? What encourages repeat business?
\item
  When looking for a new salon, how do you decide? Where do you look?
\item
  Imagine you land on the webpage of a potential new salon. What content would you expect to see?
\item
  As a new customer, would you visit the webpage with one objective, or multiple? If multiple, which would you expect to see first?
\item
  Imagine you are visiting the webpage of a salon you have been a customer of for a while. What content would you expect to see?
\item
  As a repeat customer, would you visit the webpage with one objective, or multiple? If multiple, which would you expect to see first?
\item
  What else would you want to get out of the website of a salon as a repeat customer or new customer?
\item
  Would you be more likely to visit this webpage on your phone, tablet, or desktop?
\item
  Think back to the last webpage you interacted with. In terms of navigating this webpage to accomplish your certain task/goal, describe what you liked/didn't like. What do you wish was easier?
\end{enumerate}


\section{Appendix E: Survey Analysis}

Survey can be found on \link{https://www.surveymonkey.com/r/G83YK3V}{Survey Monkey}.

\begin{enumerate}
\item
  What are you looking for in a hair salon?
  \begin{itemize}
  \item
    Cleanliness, friendliness, experience, relaxing, inviting, professionalism
  \item
    Price, customer service, location
  \end{itemize}
\item
  What helps one salon stand out from another? What improves your experience? What encourages repeat business?
  \begin{itemize}
  \item
    Rapport, listening to their ideas, customer service, respect, contact (appt confirmation). Making felt like you are not rushed.
  \item
    Skill level, quality hair cuts/styling/coloring
  \end{itemize}
\item
  When looking for a new salon, how do you decide? Where do you look?
  \begin{itemize}
  \item
    Location, online, reviews (Google, Yelp), suggestions from friends/family
  \item
    Social media presence. Pictures on Facebook and Instagram.
  \end{itemize}
\item
  Imagine you land on the webpage of a potential new salon. What content would you expect to see?
  \begin{itemize}
  \item
    Portfolio, types of products/services, bio of stylists, prices, online booking, free consultation, blogs, hours, contact info
  \item
    Business history
  \end{itemize}
\item
  As a new customer, would you visit the webpage with one objective, or multiple? If multiple, which would you expect to see first?
  \begin{itemize}
  \item
    Price, services, products, location, contact, book an appointment
  \end{itemize}
\item
  Imagine you are visiting the webpage of a salon you have been a customer of for a while. What content would you expect to see?
  \begin{itemize}
  \item
    Appointments, portfolio, services, pictures of salon, bios of stylists, reviews, blog, prices, new offers
  \end{itemize}
\item
  As a repeat customer, would you visit the webpage with one objective, or multiple? If multiple, which would you expect to see first?
  \begin{itemize}
  \item
    Most said one - booking an appt, similar to previous answer for #5 and #4.
  \end{itemize}
\item
  What else would you want to get out of the website of a salon as a repeat customer or new customer?
  \begin{itemize}
  \item
    Coupons, sale info, review, mission statements, core values
  \end{itemize}
\item
  Would you be more likely to visit this webpage on your phone, tablet, or desktop?
  \begin{itemize}
  \item
    All but 1 said phone
  \end{itemize}
\item
  Think back to the last webpage you interacted with. In terms of navigating this webpage to accomplish your certain task/goal, describe what you liked/didn't like. What do you wish was easier?
  \begin{itemize}
  \item
    Website responsiveness (my phrasing)
  \item
    Menu with a list of options
  \item
    Was too complicated, aesthetically displeasing
  \item
    Pop up prompts to sign up - did not like
  \item
    Would have liked it to be easier to navigate and use - nav across the top
  \end{itemize}
\end{enumerate}


\section{Appendix F: Potential bias}

\begin{itemize}
\item
  \textbf{Confirmation bias} involves confirming preconceived notions in your hypothesis. This can be avoided by looking for signs the original hypothesis is incorrect, testing data empirically, and involving multiple people in the needfinding process.
\item
  \textbf{Observer bias} can create bias around what we want the user to do or assist them in accomplishing a task. This can also occur in phrasing survey questions in a bias manner. This can be avoided by separating motives and experiments, scripting interactions with users, and using a third party to reviewing interview scripts and surveys.
\item
  \textbf{Social desirability bias} occurs when a participant feels obligated to give positive feedback. An easy way to avoid this is to hide the true motive of a question in the survey and recording objective data
\item
  \textbf{Voluntary response bias} addresses the fact that strongly opinionated users tend to give voluntary feedback. To avoid this, do not expose the user to survey content prematurely.
\item
  \textbf{Recall bias} highlights the fact the users are not great at recalling exactly what they did, why they did, or how they felt while doing something. This can be reduced by having users think out loud during activities or recording interview/survey data during the experiment itself.
\end{itemize}



\end{document}

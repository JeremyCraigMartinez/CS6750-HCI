\input{ main-style }

\title{Assignment M1\\}

\authoremail{Jeremy Martinez}{jmartinez91@gatech.edu}

\begin{document}
\maketitle
\thispagestyle{fancy}

% You may include a new tex file here using:
% \input{<file_name>.tex}

\begin{abstract}
I will be focusing on a website’s navigation bar. The nav bar is usually the main call to action when visiting a homepage of a web application and provides context for what actions are available. I will compare navigability between desktop/mobile versions of a website as well.
\end{abstract}


\section{Needfinding - Think Aloud}

\paragraph{Summarizing the results}
Summarize the results of the think aloud needfinding exercise here.
Reference appendix for script that was read to users on their privacy:
Reference appendix here as well where raw notes of the users perception of the interface is transcribed.

\paragraph{Summarizing the takeaways}
Summarize the takeaways from this particular plan

\paragraph{Avoiding bias}
Bias the is often encountered during think aloud exercises is that it forces the user to focus on their metacognition as they conduct the activity. This changes the organic experience and can potentially not reflect how many users interact with an interface. This will be taken into consideration when evaluating certain cognitive tasks. Also, whenever possible, the user will be asked to reflect (immediately) on a task as opposed to during the task. This will also help reduce recall bias.

\begin{itemize}
\item
  \textbf{Confirmation bias} involves confirming preconceived notions in your hypothesis. This can be avoided by looking for signs the original hypothesis is incorrect, testing data empirically, and involving multiple people in the needfinding process.
\item
  \textbf{Observer bias} can create bias around what we want the user to do or assist them in accomplishing a task. This can also occur in phrasing survey questions in a bias manner. This can be avoided by separating motives and experiments, scripting interactions with users, and using a third party to reviewing interview scripts and surveys.
\end{itemize}

\clearpage

\begin{itemize}
\item
  \textbf{Social desirability bias} occurs when a participant feels obligated to give positive feedback. An easy way to avoid this is to hide the true motive of a question in the survey and recording objective data
\item
  \textbf{Voluntary response bias} addresses the fact that strongly opinionated users tend to give voluntary feedback. To avoid this, do not expose the user to survey content prematurely.
\item
  \textbf{Recall bias} highlights the fact the users are not great at recalling exactly what they did, why they did, or how they felt while doing something. This can be reduced by having users think out loud during activities or recording interview/survey data during the experiment itself.
\end{itemize}


\section{Existing user interfaces}

\paragraph{Summarizing the results}
For this needfinding exercise, I will focus on two hair salons in the Vancouver, WA area (which is the location of the hair salon this study is for) as well as web app interface from an entirely separate industry. The interface of concern is their business website. The reason for analyzing these two subsets is to see what the local business do well to establish an intimate relationship with their neighborhood, while also recognizing what enables the chain salons to work at scale. I will discover all of these salons profiles via Google search, and more specifically, the website button next to their Google maps profile. When comparing these two interfaces, I am looking for the following attributes
\begin{itemize}
\item
  \textbf{Content}. What content are they serving? How is this content relevant to their intended audience? This will be good to cross reference with results from our survey
\item
  \textbf{Landing page}. What do they deem most important to show their customers immediately? Is this useful? Are there obvious calls to action to find more information?
\item
  \textbf{Calls to action}. How are these (if any) calls to action displayed? How easy is it to use on a desktop browser? Mobile? Tablet?
\item
  \textbf{Content per tab}. Does the nav bar do a good job of displaying enough, but not too much information? If it's too much, how could it be easier to navigate?
\end{itemize}

\subsection{Spoiled Spa and Salon}

\paragraph{Content & landing page}
for this salon serves a nav at the upper-right part of the screen for desktop and a hamburger at the upper-left for tablet/mobile. The tabs include services, gift certificates, products, specials, policies, and appointment bookings. The content of the page below contains contact information, calls to action (buttons) on their most common services, and a upsell for gift card certificates..

\paragraph{Calls to action}
for the services provided all have consistent layouts. They list out their services, tiers of each services, and prices associated with these services.

\paragraph{Content per tab}
The services tab from the nav bar has a sub nav matching the services on the landing page. The gift certificates tab has an grid of tiles all upselling gift certificates. They use a service calls Vagaro to fulfill these. The products tab allows you to add hair products to a cart (also backed by Vagaro). Specials tab lists off coupons for the various services they provide. Policies serves like an FAQ to cancellation policies, age restrictions, etc. Book an appointment has a easy to use interface listing out services and available times for appointments (also backed by Vagaro).

\subsection{Firefly Atelier Hair Salon}

\paragraph{Content & landing page}
for this salon serves a nav at the upper-right part of the screen for desktop and a hamburger at the upper-left for tablet/mobile. The tabs include home, our team, products, services, contact, promotional and events. The content of the page below contains a video on loop showing the salon on an average day giving insight to their day-to-day operations. As you scroll down, the landing page (home page) also has three calls to action (in line with the nav tabs) for services, meet our team, and contact.

\paragraph{Calls to action}
on the video of the salon there is a downward arrow indicating to check out the content below. They have a short description. The three calls to action map to the three tab in the nav bar.

\paragraph{Content per tab}
The products tab here simply lays out what products they offer/use (pictures only). There is no call to action on this page. The services tab lists out what they offer in terms of services. Each service has it's own page listing out the prices for that particular service. The contact form lists out contact information, GPS maps of their location, and an email form. "Who we are" contains a YouTube video introducing the team that works there

\subsection{Twitch.tv}
For a bit of background, Twitch is an online video game streaming platform.

\paragraph{Content & landing page}
for this web application contains a nav at the top bar, which remains the same in tablet and mobile, however, to conserve space, they reduce the tab to icons. The nav includes three tabs in the upper left (most popular three) discover, browse, and type prime, accompanied with a ... symbol to indicate all other actions. Then it includes a search bar in the center. Finally, it includes a login/signup tab in the upper right (we will ignore this for now since login/signup is not relevant to our use case).

\paragraph{Calls to action}
on the home page are to select different video game streams that are currently happening. This is to appease to their most common user. As you scroll, you just find more and more types of streams.

\paragraph{Content per tab}
The browse button takes you back to the same page (essentially a home page button). The browse button takes you a page focused around categorizing streams and enabling the user to easily sift through their wide array of content. Twitch prime is an upsell for their paid-for service.

\subsection{Summarizing the takeaways}
The salon interfaces are useful becasue they give us a persepctive of what this user type will expect to see. They will be use to seeing content similar to their competitors. However, they are narrow minding. Twitch offers an interesting perspective on the implementation. They have offered a few easy-to-find tabs in the nav for 99% of their users. Then for everyone else, they've provided an "advanced" option where they can find less popular options. This is a useful strategy that we can leverage to appease the many, while still making functionality available to the few.

\paragraph{Avoiding bias}
Tunnel vision can occur during this needfinding exercise because one can focus too much on replicating their competitors. While I believe the competitors approach contains useful parts, we will try to only retain the content they contain, and approach the navigation aspect from an entirely new perspective. The goal is to get the user to feel the most comfortable in ther interface. We are going to pay extra attention to not restrict ourselves to this style of navigation implementation if a better one exists.


\begin{itemize}
\item
  \textbf{Confirmation bias} involves confirming preconceived notions in your hypothesis. This can be avoided by looking for signs the original hypothesis is incorrect, testing data empirically, and involving multiple people in the needfinding process.
\item
  \textbf{Observer bias} can create bias around what we want the user to do or assist them in accomplishing a task. This can also occur in phrasing survey questions in a bias manner. This can be avoided by separating motives and experiments, scripting interactions with users, and using a third party to reviewing interview scripts and surveys.
\item
  \textbf{Social desirability bias} occurs when a participant feels obligated to give positive feedback. An easy way to avoid this is to hide the true motive of a question in the survey and recording objective data
\item
  \textbf{Voluntary response bias} addresses the fact that strongly opinionated users tend to give voluntary feedback. To avoid this, do not expose the user to survey content prematurely.
\item
  \textbf{Recall bias} highlights the fact the users are not great at recalling exactly what they did, why they did, or how they felt while doing something. This can be reduced by having users think out loud during activities or recording interview/survey data during the experiment itself.
\end{itemize}


\section{Needfinding - Survey}

\paragraph{Summarizing the results}
Summarize the results of the think aloud needfinding exercise here.
Reference appendix for script that was read to users on their privacy:
Reference appendix here as well for the survey
Reference appendix here for survey analysis

\paragraph{Summarizing the takeaways}
Summarize the takeaways from this particular plan

\paragraph{Avoiding bias}
To avoid bias, we will ensure this survey is anonymized and that the user is aware of that so that they answer honestly. Having this survey online can ensure we distribute it to a large audience. We will put emphasis on the focus points listed above.

\begin{itemize}
\item
  \textbf{Confirmation bias} involves confirming preconceived notions in your hypothesis. This can be avoided by looking for signs the original hypothesis is incorrect, testing data empirically, and involving multiple people in the needfinding process.
\item
  \textbf{Observer bias} can create bias around what we want the user to do or assist them in accomplishing a task. This can also occur in phrasing survey questions in a bias manner. This can be avoided by separating motives and experiments, scripting interactions with users, and using a third party to reviewing interview scripts and surveys.
\item
  \textbf{Social desirability bias} occurs when a participant feels obligated to give positive feedback. An easy way to avoid this is to hide the true motive of a question in the survey and recording objective data
\item
  \textbf{Voluntary response bias} addresses the fact that strongly opinionated users tend to give voluntary feedback. To avoid this, do not expose the user to survey content prematurely.
\item
  \textbf{Recall bias} highlights the fact the users are not great at recalling exactly what they did, why they did, or how they felt while doing something. This can be reduced by having users think out loud during activities or recording interview/survey data during the experiment itself.
\end{itemize}


\section{Data inventory}
Intro on data inventory

\paragraph{Our expectation on user types}
The intended user for this application is adults in general. However, the current customer base will be the heavy focus of whom this experience will be designed for, which includes men and women ages 30-60. We anticipate the majority of the target audience to be older than millennials.

When evaluating the intended audience, the following is taken into account

\begin{itemize}
\item
  \textbf{Demographic of our audience} fits (for the most part) men and women aged 30-60. We are choosing this because, for the specific salon owner I am working with, their client-base has matched this for the last 20 years.
\item
  \textbf{Their level of expertise} will be assumed to be not very technical. This audience will be in the grey area between a millennial and geriatric. We will assume for this reason that the user does not have a strong grasp on the differences between types of browsers, versions of browsers, or experiences between mobile vs desktop browsers. The UI should be styled in such a way that the look and feel is familiar, yet clearly geared to be optimized on the platform at hand.
\item
  \textbf{Their motivation for engaging in the task} will not be singular. The number of reasons one might visit a hair salons website could be for general salon information, hours of operation, salon team members, products/services they offer, schedule an appointment, contact information. We can try to anticipate our users needs; however, we must assume they have any one of these intentions. Therefore, having an intuitive navigation inside the application is crucial to naturally enabling the user to accomplish the task they came to do without any friction.
\end{itemize}

\subsection{Who are the users?}
What we found regarding this. What each needfinding exercise told us and how it supports our findings.

\subsection{Where are the users?}
What we found regarding this. What each needfinding exercise told us and how it supports our findings.

\subsection{What is the context of the task?}
What we found regarding this. What each needfinding exercise told us and how it supports our findings.

\subsection{What are their goals?}
What we found regarding this. What each needfinding exercise told us and how it supports our findings.

\subsection{What do they need?}
What we found regarding this. What each needfinding exercise told us and how it supports our findings.

\subsection{What are their tasks?}
What we found regarding this. What each needfinding exercise told us and how it supports our findings.

\subsection{What are their subtasks?}
What we found regarding this. What each needfinding exercise told us and how it supports our findings.


\section{Defining Requirements}
What are the requirements of your interface in terms of questions like functionalities it must provide, learnability goals it must meet, or accessibility standards it must support (as well as others)?

What metrics or criteria would you use to evaluate the success of a prototype that attempts to address these requirements?

Depending on your results and your project, you will likely emphasize some requirements over others; for example, if you are focusing on novice users, learnability will likely take a higher priority, whereas if you are focusing on experts, you may care more about efficiency.


\section{Continued Needfinding}
Briefly outline the next iteration of needfinding in which you might engage based solely on this initial experience. What remaining questions are there that would benefit from additional needfinding investigation? What new questions arose during this initial round of needfinding? What types of exercises would you do next to address these remaining or new questions?


\section{Appendix A: data protection & privacy}

\begin{itemize}
\item
  \textbf{explain the goal} of the exercise
\item
  \textbf{Explain what we'll be doing} and the sub-tasks involved
\item
  \textbf{Ask the user} if they are okay with me recording the exercise and state that it will only be used for my reflection as a means of extensive note taking
\item
  \textbf{Explain how the data} will be stored and the time to live (ttl).
\item
  \textbf{All of their information will be anonmized} and there will be nothing to correlate any data back to a specific participant. Their email/contact information will not be shared.
\item
  \textbf{Ensure they are aware} the study can be halted at any time if they so choose
\item
  \textbf{Offer to share the final deliverable} with them once the study and product is complete
\end{itemize}


\section{Appendix B: Think Aloud Notes}
Add notes here


\section{Appendix C: Survey}
\begin{enumerate}
\item
  What are you looking for in a hair salon/barber?
\item
  What helps one salon/barber stand out from another? What improves your experience? What encourages repeat business?
\item
  When looking for a new salon/barber, how do you decide? Where do you look?
\item
  Imagine you land on the webpage of a potential new salon/barber. What content would you expect to see?
\item
  As a new customer, would you visit the webpage with one objective, or multiple? If multiple, which would you expect to see first?
\item
  Imagine you are visiting the webpage of a salon/barber you have been a customer of for a while. What content would you expect to see?
\item
  As a repeat customer, would you visit the webpage with one objective, or multiple? If multiple, which would you expect to see first?
\item
  What else would you want to get out of the website of a salon/barber as a repeat customer or new customer?
\item
  Would you be more likely to visit this webpage on your phone, tablet, or desktop?
\item
  Think back to the last webpage you interacted with. In terms of navigating this webpage to accomplish your certain task/goal, describe what you liked/didn't like. What do you wish was easier?
\end{enumerate}


\section{Appendix D: Survey Analysis}
The survey analysis


\end{document}

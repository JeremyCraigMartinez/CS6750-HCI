\input{ main-style }

\title{Assignment M1\\}

\authoremail{Jeremy Martinez}{jmartinez91@gatech.edu}

\begin{document}
\maketitle
\thispagestyle{fancy}

% You may include a new tex file here using:
% \input{<file_name>.tex}

\begin{abstract}
I will be focusing on a website’s navigation bar. The nav bar is usually the main call to action when visiting a homepage of a web application and provides context for what actions are available. I will compare navigability between desktop/mobile versions of a website as well.
\end{abstract}


\section{Needfinding - Think Aloud}

\paragraph{Summarizing the results}
Summarize the results of the think aloud needfinding exercise here.
Reference appendix for script that was read to users on their privacy:
Reference appendix here as well where raw notes of the users perception of the interface is transcribed.

\paragraph{Summarizing the takeaways}
Summarize the takeaways from this particular plan

\paragraph{Avoiding bias}
Bias the is often encountered during think aloud exercises is that it forces the user to focus on their metacognition as they conduct the activity. This changes the organic experience and can potentially not reflect how many users interact with an interface. This will be taken into consideration when evaluating certain cognitive tasks. Also, whenever possible, the user will be asked to reflect (immediately) on a task as opposed to during the task. This will also help reduce recall bias.

\begin{itemize}
\item
  \textbf{Confirmation bias} involves confirming preconceived notions in your hypothesis. This can be avoided by looking for signs the original hypothesis is incorrect, testing data empirically, and involving multiple people in the needfinding process.
\item
  \textbf{Observer bias} can create bias around what we want the user to do or assist them in accomplishing a task. This can also occur in phrasing survey questions in a bias manner. This can be avoided by separating motives and experiments, scripting interactions with users, and using a third party to reviewing interview scripts and surveys.
\end{itemize}

\clearpage

\begin{itemize}
\item
  \textbf{Social desirability bias} occurs when a participant feels obligated to give positive feedback. An easy way to avoid this is to hide the true motive of a question in the survey and recording objective data
\item
  \textbf{Voluntary response bias} addresses the fact that strongly opinionated users tend to give voluntary feedback. To avoid this, do not expose the user to survey content prematurely.
\item
  \textbf{Recall bias} highlights the fact the users are not great at recalling exactly what they did, why they did, or how they felt while doing something. This can be reduced by having users think out loud during activities or recording interview/survey data during the experiment itself.
\end{itemize}


\section{Existing user interfaces}

\paragraph{Summarizing the results}
Summarize the results of the think aloud needfinding exercise here.
Reference appendix for screenshots of existing user interfaces

\paragraph{Summarizing the takeaways}
Summarize the takeaways from this particular plan

\paragraph{Avoiding bias}
Bias the is often encountered during think aloud exercises is that it forces the user to focus on their metacognition as they conduct the activity. This changes the organic experience and can potentially not reflect how many users interact with an interface. This will be taken into consideration when evaluating certain cognitive tasks. Also, whenever possible, the user will be asked to reflect (immediately) on a task as opposed to during the task. This will also help reduce recall bias.

\begin{itemize}
\item
  \textbf{Confirmation bias} involves confirming preconceived notions in your hypothesis. This can be avoided by looking for signs the original hypothesis is incorrect, testing data empirically, and involving multiple people in the needfinding process.
\item
  \textbf{Observer bias} can create bias around what we want the user to do or assist them in accomplishing a task. This can also occur in phrasing survey questions in a bias manner. This can be avoided by separating motives and experiments, scripting interactions with users, and using a third party to reviewing interview scripts and surveys.
\item
  \textbf{Social desirability bias} occurs when a participant feels obligated to give positive feedback. An easy way to avoid this is to hide the true motive of a question in the survey and recording objective data
\item
  \textbf{Voluntary response bias} addresses the fact that strongly opinionated users tend to give voluntary feedback. To avoid this, do not expose the user to survey content prematurely.
\item
  \textbf{Recall bias} highlights the fact the users are not great at recalling exactly what they did, why they did, or how they felt while doing something. This can be reduced by having users think out loud during activities or recording interview/survey data during the experiment itself.
\end{itemize}


\section{Needfinding - Survey}

\paragraph{Summarizing the results}
Summarize the results of the think aloud needfinding exercise here.
Reference appendix for script that was read to users on their privacy:
Reference appendix here as well for the survey
Reference appendix here for survey analysis

\paragraph{Summarizing the takeaways}
Summarize the takeaways from this particular plan

\paragraph{Avoiding bias}
To avoid bias, we will ensure this survey is anonymized and that the user is aware of that so that they answer honestly. Having this survey online can ensure we distribute it to a large audience. We will put emphasis on the focus points listed above.

\begin{itemize}
\item
  \textbf{Confirmation bias} involves confirming preconceived notions in your hypothesis. This can be avoided by looking for signs the original hypothesis is incorrect, testing data empirically, and involving multiple people in the needfinding process.
\item
  \textbf{Observer bias} can create bias around what we want the user to do or assist them in accomplishing a task. This can also occur in phrasing survey questions in a bias manner. This can be avoided by separating motives and experiments, scripting interactions with users, and using a third party to reviewing interview scripts and surveys.
\item
  \textbf{Social desirability bias} occurs when a participant feels obligated to give positive feedback. An easy way to avoid this is to hide the true motive of a question in the survey and recording objective data
\item
  \textbf{Voluntary response bias} addresses the fact that strongly opinionated users tend to give voluntary feedback. To avoid this, do not expose the user to survey content prematurely.
\item
  \textbf{Recall bias} highlights the fact the users are not great at recalling exactly what they did, why they did, or how they felt while doing something. This can be reduced by having users think out loud during activities or recording interview/survey data during the experiment itself.
\end{itemize}


\section{Data inventory}
Intro on data inventory

\paragraph{Our expectation on user types}
The intended user for this application is adults in general. However, the current customer base will be the heavy focus of whom this experience will be designed for, which includes men and women ages 30-60. We anticipate the majority of the target audience to be older than millennials.

When evaluating the intended audience, the following is taken into account

\begin{itemize}
\item
  \textbf{Demographic of our audience} fits (for the most part) men and women aged 30-60. We are choosing this because, for the specific salon owner I am working with, their client-base has matched this for the last 20 years.
\item
  \textbf{Their level of expertise} will be assumed to be not very technical. This audience will be in the grey area between a millennial and geriatric. We will assume for this reason that the user does not have a strong grasp on the differences between types of browsers, versions of browsers, or experiences between mobile vs desktop browsers. The UI should be styled in such a way that the look and feel is familiar, yet clearly geared to be optimized on the platform at hand.
\item
  \textbf{Their motivation for engaging in the task} will not be singular. The number of reasons one might visit a hair salons website could be for general salon information, hours of operation, salon team members, products/services they offer, schedule an appointment, contact information. We can try to anticipate our users needs; however, we must assume they have any one of these intentions. Therefore, having an intuitive navigation inside the application is crucial to naturally enabling the user to accomplish the task they came to do without any friction.
\end{itemize}

\subsection{Who are the users?}
What we found regarding this. What each needfinding exercise told us and how it supports our findings.

\subsection{Where are the users?}
What we found regarding this. What each needfinding exercise told us and how it supports our findings.

\subsection{What is the context of the task?}
What we found regarding this. What each needfinding exercise told us and how it supports our findings.

\subsection{What are their goals?}
What we found regarding this. What each needfinding exercise told us and how it supports our findings.

\subsection{What do they need?}
What we found regarding this. What each needfinding exercise told us and how it supports our findings.

\subsection{What are their tasks?}
What we found regarding this. What each needfinding exercise told us and how it supports our findings.

\subsection{What are their subtasks?}
What we found regarding this. What each needfinding exercise told us and how it supports our findings.


\section{Defining Requirements}
What are the requirements of your interface in terms of questions like functionalities it must provide, learnability goals it must meet, or accessibility standards it must support (as well as others)?

What metrics or criteria would you use to evaluate the success of a prototype that attempts to address these requirements?

Depending on your results and your project, you will likely emphasize some requirements over others; for example, if you are focusing on novice users, learnability will likely take a higher priority, whereas if you are focusing on experts, you may care more about efficiency.


\section{Continued Needfinding}
Briefly outline the next iteration of needfinding in which you might engage based solely on this initial experience. What remaining questions are there that would benefit from additional needfinding investigation? What new questions arose during this initial round of needfinding? What types of exercises would you do next to address these remaining or new questions?


\section{Appendix A: data protection & privacy}

\begin{itemize}
\item
  \textbf{explain the goal} of the exercise
\item
  \textbf{Explain what we'll be doing} and the sub-tasks involved
\item
  \textbf{Ask the user} if they are okay with me recording the exercise and state that it will only be used for my reflection as a means of extensive note taking
\item
  \textbf{Explain how the data} will be stored and the time to live (ttl).
\item
  \textbf{All of their information will be anonmized} and there will be nothing to correlate any data back to a specific participant. Their email/contact information will not be shared.
\item
  \textbf{Ensure they are aware} the study can be halted at any time if they so choose
\item
  \textbf{Offer to share the final deliverable} with them once the study and product is complete
\end{itemize}


\section{Appendix B: Think Aloud Notes}
Add notes here


\section{Appendix C: Survey}
The survey


\section{Appendix D: Survey Analysis}
The survey analysis


\end{document}

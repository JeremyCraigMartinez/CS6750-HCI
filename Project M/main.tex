\input{ main-style }

\title{a new project\\}

\authoremail{Jeremy Martinez}{jmartinez91@gatech.edu}

\begin{document}
\maketitle
\thispagestyle{fancy}

\clearpage

\subsection{Think-Aloud}

In the think-aloud needfinding exercise, our participants will be read a script to set context, given a tasks, sub-tasks and a goal to accomplish, and a tool (interface) with which to complete their goal. The products being analyzed consist of Google Maps\footnote{https://maps.google.com}, Furkot\footnote{https://www.furkot.com}, and RoadTripper\footnote{https://www.roadtrippers.com}.

All of these interfaces will be mobile only. We will not focus on desktop because we expect our users to interact with the tool during their trip, which is more realistic to do with a mobile device. Of these three products, Furkot is the only one without a mobile app in the iOS app store. For this reason, we will use their website on a mobile browser.

\paragraph{Needfinding Plan} the participant will be given the following instruction set as pretext to the needfinding exercise. The participant will also be instructed to articulate their though process as best they can along the way.

\begin{itemize}
\item
  You are panning a multi-day road-trip along the famous Route 66 highway in the United States. This will begin in Chicago, IL and end in Santa Monica, CA.
\item
  You are expected to allocate no more than 7 days for this road trip.
\item
  Your tasks will include
  \begin{itemize}
  \item
    Find expected overall trip distance
  \item
    Find expected overall trip time (driving only)
  \item
    Find checkpoints (up to 7 nights maximum, with the last being Santa Monica) for sleeping along route.
  \item
    Find five tourist attractions along route
  \item
    Find (for brevity) one place to eat along route for each day of the trip
  \end{itemize}
\item
  Your final goal will be to have a clear road trip itinerary documenting day-by-day what you'll be doing and where you'll need to be.
\end{itemize}

This script will be broken down by task. Before each task we will ask the user to pay attention to how easy it is to navigate the app to accomplish their task, how clear the operators are, reflect on if they were successful, if they got the result they intended, what could have been added/removed to make the task easier/possible, how the experience compares/contrasts with their expectations, etc.

\paragraph{Data Inventory} the participant demographic will have to span the data inventory mapped out in the introduction. Both male and female participants, ages 16-56 (arbitrary ceiling here), and spanning multiple ethnicity's. We will also want to expose the app to individuals as well as groups. We anticipate planning a road-trip to be, at times, a collective effort. We will also want to form questioning to highlight design expectations between novice travelers and avid travelers.

\paragraph{Potential Bias} some typical biases with think-aloud needfinding exercises is observer bias. Giving the user a script of intentional tasks to accomplish can distort their thinking. It may make the user more programmatic in how they approach the tool. However, we do this in order to isolate and control the variables of all the other goals/tasks a user may have when using this tool. We will also vet the script to ensure it does not coach the user on how to accomplish these tasks/sub-tasks. Recall bias can be a concern as well. In fact, think-aloud exercises are design to account for this, to prevent users from their inability to recall motives from an earlier exercise. Asking a user to articulate their thought process as they do something can cause them to act more consciously.

One other potential bias we may encounter during this exercise is an economic bias. Lower budget travel alternatives may be preferable for some/most participants (attractions that cost little to no money). If one of these tools only suggests attractions that are expensive, this could reflect as a negative user experience. One possible way to avoid this would be to allocate a budget to each participant (however, this would add a large amount of complexity).

\end{document}

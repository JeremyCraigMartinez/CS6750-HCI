\input{ main-style }

\title{Assignment 1\\}

\authoremail{Jeremy Martinez}{jmartinez91@gatech.edu}

\begin{document}
\maketitle

\begin{abstract}
Multiple interfaces are observed from three particular frames of mind: the processor model, predictor model, and participant model. The processor model evaluates an interfaces in terms of whether the core functionality is achievable. The predictor model evaluates an interfaces intuition and is measured by one's ability to predict aspects of the interface. The participant model understands the interface in broader contexts and considers third party factors that may influence perception/ability to complete a task
\end{abstract}

\section{Question 1}
When evaluating interfaces based on the processor model, predictor model, and participant model, Piazza stands out as a tool that does some things well, but could gain immensly from putting more emphasis on the other two. Piazza is a tool that is designed to act as the classroom portion of a massive open online course (MOOC). Piazza has two main components to it's platform, the sidebar and the main thread which highlights whatever discussion you've selected from the sidebar. It has two main components to it's platform, the sidebar and the main thread which highlights whatever discussion you've selected from the sidebar (defaulting to some discussion if none selected). In order to achieve this the tool must address the following, among other things,

\begin{itemize}
\item
  Facilitate discussion on relevant course topics
\item
  Provide a platform for a student to interact with their instructors/teaching assistants
\item
  Provide a medium in which the instructor can make class-wide announcements
\end{itemize}

\subsection{Piazza from a processor model}

The processor model treats the user as another (computer) processor in tandem with the interface.

\clearpage

This asks objective questions on functionality like whether a user is able to complete a given task, and in what amount of time or how effectively.

An interface that does well in the processor model will fit within human limits. In terms of an interface, it is physically visible and comprehendable to the user. This is evaluated with quantitative measures, like whether they can complete some task and how quickly? If A/B testing, with what percentage of completion does one method (method A) have over another (method B)?

Objectively speaking, Piazza accomplishes the task of hosting a virtual classroom because all of the functionality is there.

\begin{itemize}
\item
  The main thread facilitates discussion on whatever thread the student is concerned with. However, this discussion does not happen in real time, and so engaging in a debate-like conversation is lost in this platform, or is drawn out over several hours/days.
\item
  The student has the ability to post questions/discussions/polls/etc. for the entire student body, only innstructors, or only teaching assistants.
\item
  Piazza has a priority queue that only the instructor is allowed to post in which serves as a logically separated area for class-wide announcements
\end{itemize}

\subsection{Piazza from a predictor model}

The predictor model takes into account cognition of the user of the interface and how well they are able to predict aspects of the interface. This evaluates how well an individual can predict what the outcome of an action will be, whether that outcome reflects what you were expecting, and if the next call to action follows intuition. This involves getting inside the users head, what they're thinking, what they're seeing, and feeling during a given task. Measuring an interfaces success at this often entails what a user-researcher might evaluate on a UI/UX design team.

An interfaces that does well in the predictor model will map users inputs to outputs in an intuitive way. The interface must fit with the knowledge of the user. This means, it must help the user learn what they are trying to learn, and efficiently leverage what they do already know. Contrary to the processor model, we evaluate this with qualitative results

\begin{itemize}
\item
  It is not clear to a user if/when a user will respond to their post, and it is difficult for the user to predict when they will be able to engage in that discussion (having been responded to). There is not an efficient alerting system with this either.
\item
  The interface in which to post a new discussion is intuitive, not too cluttered, and easy to edit making it straightforward to use. Piazzas interface with this task matches existing discussion platforms, which leverage the users existing knowledge well.
\item
  A user will take the path of least resistance, so having higher priority discussion (announcements from the instructor) always queued to the top allows the user to predict what they should view first and foremost.
\end{itemize}

\subsection{Processor model vs predictor model}

From a purely functional standpoint (processor model), Piazza does many things well. The real time discussion is a limiting factor, and so adding a feature or required online period could address this. From a predictor model argument, enhancements to this feature would include a more rich discussion environment, which I will use Slack as an example

\begin{itemize}
\item
  Adding profiles of students in another sidebar with some indication that they are online would allow students to leverage real-time discussion with students they can see are currently online
\item
  Adding thought bubbles/typing bubbles to show another student is in the process of responding would allow the user to anticipate when their question will get a response
\item
  Giving the user some indication that their posts have received activity (intelligent alert system) would enhance these discussions as well
\end{itemize}

While some of the discussion above in regard to the predictor model discusses other students, I want to emphasize that this differs from a participant model, since I am only referring to a user's interaction with the discussion board interface. The discussion board inherently encapsulates a response to a users post, so it is impossible to contemplate the discussion board without considering another post in which the user is responding to.

\section{Question 2}

\subsection{Rosetta Stone}

Rosetta Stone is a web-based application for learning a foreign language. They're primary platforms are designed for desktop browsers and a mobile application. Contexts that this tool is used in include, but are not limited to,

\begin{itemize}
\item
  Non-distracted modern browser use
\item
  Non-distracted legacy browser use
\item
  Distracted browser (both modern and legacy) use
\item
  Distracted mobile use
\item
  Non-distracted mobile use
\end{itemize}

\subsection{Context analysis}

\paragraph{Non-distracted modern browser}
Perhaps the optimal context to use Rosetta Stone in terms of memory retention, speed, and ease of use would be with a modern browser without any distraction. In this context, the user has no distraction, can pay full attention to the task at hand, and maximize their retention in the vocabulary, pronunciation, grammar and speech comprehension.

One could make a similar evaluation of non-distracted mobile use in a secluded environment. While the interface will change significantly, the user will still be able to accomplish the same modules. One (and possibly the only) area the experience may be slightly less desirable would be with the writing portion and not having a keyboard at their disposal could make this cumbersome and time consuming.

\paragraph{Non-distracted legacy browser}
Slightly tweaking this the previous context to accessing Rosetta Stone via a legacy browser introduces some issues from a functionality standpoint (more toward an analysis of the processor model in this context). If one were to access this application in an old version of internet explorer, the application may not function properly rendering certain tasks unachievable. This will become even more apparent after 2020 when Firefox and Chrome cut support for Adobe Flash (the platform Rosetta Stone is written on).

\paragraph{Distracted browser/mobile}
Distracted use (whether this be mobile or desktop) introduces a problem with this task since not having one's full attention inhibits memory retention. This becomes an even larger concern when considering distracted use on mobile. When one is seated and using a laptop/desktop computer, their are fewer options for distractions. However, with a mobile phone, this can be used on the go, which invites many other forms of distractions.

\paragraph{Case study - mobile}
Consider a scenario where someone has a long commute to work (via the bus). They decide they want to spend this commute time more efficiently by learning a new language, and so they (pay for and) download the Rosetta Stone app (which has amazing reviews 5 stars with 72k reviews on the Apple app store). They start working through the modules on a bus ride and (unexpectedly) come to the speaking portion of a given module. However, they think it will look strange to start speaking outloud in a foreign language to themselves at 7 AM on the bus, and so they skip this portion of the module. If this is the only context in which they use this application, they may develop their understanding of this foreign language in a lopsided manner with their speach lagging far behind speech comprehension, reading, and writing. This UI/UX is not ideal and will result in their users not developing well rounded foreign language skills which could reflect poorly on their reviews.

\subsection{Context recognition/adaptation}

As far as context recognition is concerned, this is very easy to do, and I am sure Rosetta Stone is doing so already. This is achievable via HTTP headers, and other operating system recognition tactics that exist in most development platforms. In order to prevent their users from having a bad experience in a browser, they can gate the functionality of their application and prompt the user with a message, "please upgrade your browser version to continue." They could also attempt to support legacy browsers, however, this may require them to rewrite their platform from scratch (without Adobe Flash).

For enhancing the mobile experience, it is difficult to circumvent the speech portion, since you don't want to write it off entirely. However, they should be able to toggle a setting that says suppress these modules for now and come back to them later. As it stands now, sifting through unfinished modules and repeating sections is cumbersome and unintuitive. My guess is that many users never revisit these and just move on to the next module leaving the previous one incomplete.




\subsection{Body text}
Body text is set in the regular weight at 11 points with custom line spacing of 1.26, and 8.5 points of spacing added after each paragraph. It should be justified and paragraphs should not be indented. These styles can be automatically applied using the \emph{Normal} paragraph style.

\textbf{Bold} and \emph{italics} should be used for emphasis. Hyperlinks may be inserted in the text, as well as \texttt{in-line} code, superscripts \super{like this}, and subscripts \subsc{like this}.

\subsection{Title \& subtitle}
The paper title should be set in the regular weight at 17 points with 1.15 line spacing, centered at the top of the first page. The title may span up to three lines. For typical assignments, the document title may be as simple as “Assignment 1.” More specialized assignments may warrant more unique paper names, like “A Proposal to Create a New Document Format.”
The author’s name and email should come next unless you want to or were asked to submit anonymously, in which case this can be omitted. They should be set in the same size and weight as body text, centered. These styles can be applied using the Title and Subtitle paragraph styles.

\subsection{Headings}
Headings should all be set in the same size as body text (11 points) in bold. With the exception of \emph{Heading 1}, they should all have 8.5 points of space added before and after. They should be hierarchically numbered: Microsoft Word will do this automatically when you use the appropriate paragraph styles, but Google Docs users will need to number their headings manually.

\subsubsection{Heading 1}
Heading 1 should be set in all caps. It should have 11 points of space added before and 8.5 points of space added after.

\subsubsection{Headings 2-4}
Besides \emph{Heading 1}, which is set in caps, headings should always use sentence case (i.e., first word capitalized) rather than title case; after all, they are not titles. \emph{Heading 2} should be set in bold roman (upright), and \emph{Heading 3} should be set in bold italics. The use of headings beyond \emph{Heading 3} is discouraged.

\paragraph{Heading 4}
\emph{Heading 4} is provided as a run-in sidehead. Like \emph{Heading 3}, it is set in bold italics, but it should be followed by an em dash and flow right into the text, as seen at the beginning of the current paragraph. It should be used more as a list style than a heading style, e.g. to set off a list of principles in a heuristic evaluation.

\subsection{Page layout}
JDF uses the US Letter paper size (8.5'' x 11''). It has a top margin of 1'', and bottom and side margins of 1.5''. This yields a text block of 5.5'' x 8.5'', which is exactly $\frac{1}{2}$ the size of the page, divided lengthwise.

The page number should be included in the bottom margin, 1'' from the bottom of the page --- this creates symmetry with the top margin. No other elements should be placed in the margins.

\section{Presentational elements}
You are encouraged to use presentational elements liberally, as long as they add to the clarity of your submissions. They often require less space and fewer accompanying words to explain a given concept, and do a far better job of it.

\subsection{Figures}
Figures should always be centered on the page, although they may also take up the entire width and height of the text block. Figures should always be referenced in the text, and they should include a descriptive caption. Figures may also be equations, diagrams, or other kinds of content.

If your figure includes a white background (e.g. an interface design or graph), it may aid legibility to add a $\frac{1}{4}$ point black border.

\begin{figure}[H]
  \centering
  \includegraphics[scale=0.95]{figs/example-image2}
  \caption{Make sure your flowcharts are more useful than this one. Source: \link{https://xkcd.com/1195/}{XKCD}.}
  \label{fig::2}
\end{figure}

Figure captions should be centered beneath the corresponding figure. The label for the figure, e.g. “Figure 1,” should be bolded, and the entire caption should be 8.5 points with 14 points of line spacing. If need be, you may have one figure caption corresponding to multiple consecutive figures and use either locational descriptors (e.g. “top left,” “middle”) or labels (e.g. “A”, “B”) to map parts of the caption to parts of the figure. Make sure that caption falls on the same page as the corresponding figure or table; you may rearrange text to make this work.

\subsection{Tables}
You have freedom to format tables in the way that works best for your data. Generally, text should be left-aligned and numbers should be right-aligned or aligned at the decimal – you can do this using a \link{https://practicaltypography.com/tabs-and-tab-stops.html}{custom tab stop}. The default table style shown in \autoref{table:1} reduces the text size to be equal to the caption text.

Table captions should be formatted the same way as figure captions, but they should be placed above the table. The popular mnemonic for this is: figures at the foot, tables at the top. Like figures, tables should not exceed the margins and should be centered on the page.

\begin{table}[H]
  \centering
  \caption{Mathematical constants. Notice how the approximations align at the decimal.}
  \label{table:1}
  \begin{tabular}{@{}lcrl@{}}
    \textbf{Name} & \textbf{Symbol} & \textbf{Approximation} & \textbf{Description}\\
    \midrule
    Golden ratio & $\omega$ & 1.618 & Number such that the ratio of 1 to the number is equal\\
    & & & to the ratio of its reciprocal to 1\\
    \midrule
    Euler's number & $\epsilon$ & 2.71828 & Exponential growth constant\\
    \midrule
    Archimedes' & $\pi$ & 3.14 & The ratio between circumference and diameter of a\\
    constant & & & circle\\
    \midrule
    One hundred & A\super{+} & 100.00 & The grade we hope you'll all earn in this class
  \end{tabular}
\end{table}

\subsection{Additional elements}
There are additional elements you may want to include in your paper, such as in-line or block quotes, lists, and more. For other content types not covered here, you have reasonable flexibility in determining how it should be used in this format.

\subsubsection{Quotes}
If you would like to quote an outside source, you may do so with quotation marks followed by a citation. If a quote is fewer than three lines, you may write it in-line. It is acceptable to replace pronouns with their target in brackets for clarity. For example, “Heavy use of peer grading would compromise [the school’s] reputation” (Joyner, 2016). If a quote exceeds three lines, you should set it as its own paragraph with 0.5'' side margins, using the \emph{Blockquote} style.

\begin{quoting}
Whether or not the grades generated by peers are reliably similar to grades
generated by experts is only one factor worth considering, however. Student
perception is also an important factor. A recent study indicated that reliance
on peer grading is one of the top drivers of high MOOC dropout rates. This
problem may be addressed by reintroducing some expert grading where possible''
(Joyner 2016)
\end{quoting}

\subsubsection{Lists}
Bulleted and numbered lists are indented 0.5'' from the left margin, with the bullet or number hanging in the margin by 0.25'' (the default format).

\begin{itemize}
\item
  Here's an item.

\item
  Like numbered lists, the second line along a single line in a bulleted list is
  at the same level of indentation.
\end{itemize}

Bulleted lists follow the same format:
%
\begin{enumerate}
\item
  This is an item.
\item
  Note that the left side of the text is aligned, as are the numerals.
\item
  Notice also that a second line corresponding to the same bullet is also
  indented at the same level of the previous lines.
\end{enumerate}


\section{Procedural elements}

\notext

\subsection{In-line citations}
Articles or sources to which you refer should be cited in-line with the
authors' names and the year of publication.\footnote{In-line citations are preferred over footnotes, and we favor APA citation format for both in-line citations and reference lists. Refer to the \link{https://owl.purdue.edu/owl/research_and_citation/apa_style/apa_formatting_and_style_guide/in_text_citations_the_basics.html}{Purdue Online Writing Lab}, or follow the above examples. Footnotes should use 8.5 point text with 1.26 line spacing.
} The citation should be placed close
in the text to the actual claim, not merely at the end of the paragraph. For
example: students in the OMSCS program are older and more likely to be employed
than students in the on-campus program \citep{Joyner17}. In the event of
multiple authors, list them. For example: research finds sentiment analysis of
the text of OMSCS reviews corresponds to student-assigned ratings of the course
\citep{Newman18}. You may also cite multiple studies together. For example:
several studies have found students in the online version of an undergraduate
CS1 class performed equally with students in a traditional version
\citep{Joyner18a,Joyner18b,Joyner19}. If you would like to refer to an author in
text, you may also do so by including the year (in parentheses) after the
author's name in text. If a publication has more than 4 authors, you may list
only the first author followed by `et al' . For example:
Joyner~et~al.~(\citeyear{Joyner16}) claim that a round of peer review prior to
grading may improve graders' efficiency and the quality of feedback given. This
applies to parenthetical citations as well, e.g. \citep{Joyner16}.

\subsection{Reference lists}
References should be placed at the end of the paper in a dedicated section. Reference lists should be numbered and organized alphabetically by first author’s last name. If multiple papers have the same author(s) and year, you may append a letter to the end of the year to allow differentiated in-line text (e.g. Joyner, 2018a and Joyner, 2018b in the section above). If multiple papers have the same author(s), list them in chronological order starting with the older paper. Only works that are cited in-line should be included in the reference list. The reference list does not count against the length requirements.

\bibliographystyle{apacite}
\bibliography{references}



\clearpage
\section{Appendices}
You may optionally move certain information to appendices at the end of your paper, after the reference list. If you have multiple appendices, you should create a section with a Heading 1 of “Appendices.” Each appendix should begin with a descriptive Heading 2; appendices can thus be referenced in the body text using their heading number and description, e.g. “Appendix 5.1: Survey responses.” If you have only one appendix, you can label it with the word “Appendix” followed by a descriptive title, e.g., “Appendix: Survey responses.”

These appendices do not count against the page limit, but they should not contain any information required to answer the question in full. The body text should be sufficient to answer the question, and the appendices should be included only for you to reference or to give additional context. If you decide to move content to an appendix, be sure to summarize the content and note it in relevant place in the body text, e.g., “The raw data can be viewed in Appendix 5.1: Survey responses.”

\end{document}

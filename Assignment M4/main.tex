\input{ main-style }

\title{Assignment M4\\}

\authoremail{Jeremy Martinez}{jmartinez91@gatech.edu}

\begin{document}
\maketitle
\thispagestyle{fancy}

\begin{abstract}
I will be focusing on a website’s navigation bar. The nav bar is usually the main call to action when visiting a homepage of a web application and provides context for what actions are available. I will compare navigability between desktop/mobile versions of a website as well. See note on prototypes in appendix A.
\end{abstract}

\section{Qualitative Evaluation}

\subsection{Evaluation Plan}

This evaluation will focus on the prototype from Assignment M3 found in Appendix B. \textbf{The participants} for this evaluation will be gathered through \textemph{friends, family, and fellow students in the course}. The participants \textbf{will be recruited} through personal contact and a Piazza post in the class forum. The evaluation will \textbf{take place online}. It will be done using Post Event Protocol on a wireframing hosted at an address. The user will be directed to the website to step through the wireframing, then be prompted to take a survey. The results for the survey will be \textbf{hosted and recorded on Survey Monkey}.

\subsection{Evaluationn Content - Post Event Protocol}

\paragraph{Not that} post event protocol evaluation/needfinding is prone to recall bias. However, our wireframe prototypes are in their infancy and are trivial. For this lack of complexity, recall bias will not be a major concern with this exercise.

\subsection{What directions will the participant be given?}
\begin{itemize}
\item
  \textbf{Context}
  \begin{itemize}
  \item
    You are looking for a new hair salon/barber on your smart phone
  \item
    You Google search the area for a new salon/barber and come upon one result that takes you to Google Maps
  \item
    You click the "Website" icon in Google Maps that drops you onto the salon/barber's website in your phone's mobile browser
  \end{itemize}

\clearpage

\item
  \textbf{Your general objective} is to find information on this salon/barber in order to determine if you would be interested in giving them your business.
\end{itemize}

\subsection{What data will be gathered during their engagement?}
Since this exercise is \textbf{happening asynchronously}, it is difficult to capture very much data during the engagement. However, as an attempt to gather at least some, there will be a text area in the UI while the user is looking through the wireframe to jot down thoughts that occur to them as they view the prototype. The user will be prompted to use this textarea to think aloud. To draw emphasis to this and ensure it is not forgotten during the process, we will put some sort of CTA/emphasis/animation drawing the user's attention to take notes when there is inactivity on the page.

\subsection{What questions will you ask after they’re done?}
\begin{itemize}
\item
  What was your overall impression of this user interface?
\item
  How easy was it to find what you needed?
\item
  Was there anything you needed to find that wasn't there?
\item
  What was your impression of the search capability?
\item
  Were the initial six section relevant to you?
\item
  Were you always aware where you were and where the navigation on the page was located?
\item
  Did you ever make any mistakes in the UI or click something you did not intend to?
\item
  Was it clear how to find this business, what their hours of operation were, and how to contact them?
\end{itemize}

\subsection{Addressing the Data Inventory}
This evaluation addresses the data inventory in the following ways:

\begin{itemize}
\item
  It is concerned with button placement and slips due to accidental button presses. This addresses the \textbf{older demographic of our users} and their inability for precision in a small screen space.
\item
  Their \textbf{level of expertise} is taken into account and mitigated by potentially presenting them immediately with their first six needs. We will see how relevant those needs are for them. We will also be evaluating their ability to use the search and whether they found everything they were looking for (\textbf{what do they need?}).
\item
  The \textbf{younger demographic} is targeted by putting emphasis on the interface in a mobile phone/browser.
\item
  \textbf{Where they are} and more importantly, are they able to find us is addressed.
\item
  \textbf{The context of their task} and \textbf{goals} are provided as direction prior to the evaluation. These two aspects of our data inventory are specific to the problem we are analyzing.
\end{itemize}

Depending on the responses to the post event protocol evaluation, and how we measure these responses with the data inventory taken into consideration, we will be able to measure the effectiveness of our UI. This will provide valuable, contextual insight as to whether we are creating our UI with our data inventory in mind.

\section{Empirical Evaluation}
This evaluation will focus on the prototype from Assignment M3 found in Appendix C.

\subsection{Control and Experimental Conditions}
\paragrah{What is being tested} there are two major factors being tested in this prototype. 1. the user's awareness of lateral pages. 2. the user's ability to find the content they are most concerned with. 3. (related to the second) the gulf of execution for them to get to this content.

\paragrah{What is the point of comparison} the point of comparison will just be the user's evaluation/opinion on the prototype's effectiveness. This will not be comparing one prototype to another (which the lectures suggest you should do) since the prompt for this assignment explicitly says to analyze one prototype for the empirical evaluation.

\subsection{Null and Alternative Hypotheses}
The null hypothesis is what you assume to be true unless you can find conclusive proof for your alternative hypothesis.

\subsection{Experimental Method Used}
Will it be between-subjects or within-subjects? How will subjects be assigned to groups, what will they complete as part of their condition, and what data will they generate? What analysis will you use on this data?

\subsection{Confounding Variables}
Identify what lurking variables might confound your data.

\section{Predictive Evaluation}

\subsection{GOMS Model}
Describe the specific task or tasks that you’ll be addressing with that predictive evaluation. What will the user’s goal be? What operators will be available to them? Will you be evaluating a user accomplishing a single goal they know how to do in advance, or will you be evaluating a user’s navigation around the interface to figure out how to accomplish their goal?

\section{Preparing to execute}
Select two of these evaluations to complete for the next assignment, and explain why you selected those two.

\section{Appendix A - Permission to Use Same Prototypes}

Note for the grading teaching assistant. I mistakenly used the same type of prototype for all three prototypes in assignment M3. I checked with the profressor in Piazza (https://piazza.com/class/jze97ormuua2w2?cid=831) and they confirmed that it was okay to continue with these same three prototypes here in Assignment M3. Since the deduction for misreading the prompt and not varying my prototypes was incured in the grade for M3, it should not be considered a deduction in this assignment. I appologize for the inconvinience and thank you for the consideration.

\section{Appendix B - Prototype 1 - Qualitative Evaluation}

\begin{figure}[H]
  \centering
  \includegraphics[scale=0.07]{figs/prototype-2}
  \caption{Prototype 2 - Grid with search bar}
  \label{fig::1}
\end{figure}

\section{Appendix C - Prototype 2 - Empirical Evaluation}

\begin{figure}[H]
  \centering
  \includegraphics[scale=0.07]{figs/prototype-1}
  \caption{Prototype 1 - Fixed nav with search icon}
  \label{fig::1}
\end{figure}

\section{Appendix D - Prototype 3 - Predictive Evaluation}

\begin{figure}[H]
  \centering
  \includegraphics[scale=0.07]{figs/prototype-3}
  \caption{Prototype 3 - Carousel with search bar}
  \label{fig::1}
\end{figure}


\end{document}

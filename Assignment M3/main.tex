\input{ main-style }

\title{Assignment M3\\}

\authoremail{Jeremy Martinez}{jmartinez91@gatech.edu}

\begin{document}
\maketitle
\thispagestyle{fancy}

\begin{abstract}
I will be focusing on a website’s navigation bar. The nav bar is usually the main call to action when visiting a homepage of a web application and provides context for what actions are available. I will compare navigability between desktop/mobile versions of a website as well.
\end{abstract}


\section{Brainstorming Plan}
First, outline a brainstorming plan. Explicitly note the rules you will follow, the time you will allocate to brainstorming, and the standards you will meet before moving forward. Remember, your brainstorming plan should be characterized by objective criteria to meet defined in advance.


\section{Brainstorming Execution}
Then, execute your individual brainstorming plan and report the ideas you provided. The ideas that you provide will depend on how you approach brainstorming: you might supply a flat list of ideas, an image of your brainstorming worksheet, an organized list of alternatives, etc. You may include a picture of your brainstorming sheet as a significant portion of your deliverable for this item.


\section{Selection Criteria}
After brainstorming several alternatives, detail the selection criteria you will use to select which three ideas to move forward to prototyping. This may take the form of the rules that will be applied to selecting the alternatives to move forward, or this may take the form of an explanation of the more situated reasoning behind why certain alternatives are selected. In short, explain how the alternatives to move to prototyping either will be or were selected.
During this phase, specifically connect your selection criteria to the requirements definition or data inventory you posed in Assignment M2. For example, if one of your requirements was that the interface be affordable, then alternatives related to expensive hardware might be avoided.


\clearpage


\section{Prototype}

\begin{enumerate}
\item
  A textual prototype: a plaintext description of the idea, how it will work, what its functionality will be, etc. A textual prototype should be sufficiently detailed to get feedback. This is well-suited for many types of design alternatives.
\item
  A verbal prototype: although this would be presented in text, a verbal prototype would take the form of a loose conversation script about the questions you might ask a person, the answers you would anticipate, the branches you would plan, etc. A verbal prototype is intended to be more dynamic and interactive than a textual prototype. This is well-suited for many types of design alternatives.
\item
  A paper prototype or wireframe: a hand-drawn or simplistic wireframe of the interface you intend to create. It should be thorough enough to get user feedback on its design, but not so detailed that revision would require significant effort; after all, the goal is to get feedback. This is particularly well-suited for a desktop program, tablet app, or web site.
\item
  A card prototype: a collection of smaller screens that can be iteratively shown to a user to simulate their interaction. For example, a user might be asked what they would do when faced with one screen, and then shown a different card representing the screen that would result. This is particularly well-suited for mobile apps where there is less screen real estate.
\item
  A Wizard of Oz prototype: a script of the instructions you would give a user, the commands or actions you would anticipate them making, and the responses you would return to those corresponding actions. This is particularly well-suited to prototypes that do not involve a visible screen, like auditory interfaces.
\item
  A physical prototype: a physical object that allows you to mimic the actions a user might take with your prototype. Rather than having the prototype actually operate as intended, you would note the ease with which the user interacts with the physical device, and potentially simulate the results of different actions. This is particularly well-suited to prototypes for new hardware.
\end{enumerate}


\subsection{Prototype 1}
Note that for most of these types (text, verbal, Wizard of Oz, physical), we would expect a textual description of the prototype in the body of the paper. For a paper prototype or wireframe, we would generally expect one page of your assignment to be the prototype, with around a half-page description/explanation of the prototype following. For a card prototype, we would similarly expect a portion of the prototype in the paper, but you may need to place the rest of the prototype in an appendix.
After creating the prototype, evaluate it from the perspective of the requirements you gathered in Assignment M2. Which requirements does it meet? Which requirements does it miss? How well does the prototype mesh with the audience described in your data inventory?


\subsection{Prototype 2}
Note that for most of these types (text, verbal, Wizard of Oz, physical), we would expect a textual description of the prototype in the body of the paper. For a paper prototype or wireframe, we would generally expect one page of your assignment to be the prototype, with around a half-page description/explanation of the prototype following. For a card prototype, we would similarly expect a portion of the prototype in the paper, but you may need to place the rest of the prototype in an appendix.
After creating the prototype, evaluate it from the perspective of the requirements you gathered in Assignment M2. Which requirements does it meet? Which requirements does it miss? How well does the prototype mesh with the audience described in your data inventory?


\subsection{Prototype 3}
Note that for most of these types (text, verbal, Wizard of Oz, physical), we would expect a textual description of the prototype in the body of the paper. For a paper prototype or wireframe, we would generally expect one page of your assignment to be the prototype, with around a half-page description/explanation of the prototype following. For a card prototype, we would similarly expect a portion of the prototype in the paper, but you may need to place the rest of the prototype in an appendix.
After creating the prototype, evaluate it from the perspective of the requirements you gathered in Assignment M2. Which requirements does it meet? Which requirements does it miss? How well does the prototype mesh with the audience described in your data inventory?

\end{document}

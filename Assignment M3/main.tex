\input{ main-style }

\title{Assignment M3\\}

\authoremail{Jeremy Martinez}{jmartinez91@gatech.edu}

\begin{document}
\maketitle
\thispagestyle{fancy}

\begin{abstract}
I will be focusing on a website’s navigation bar. The nav bar is usually the main call to action when visiting a homepage of a web application and provides context for what actions are available. I will compare navigability between desktop/mobile versions of a website as well.
\end{abstract}


\section{Brainstorming Plan}
The brainstorming plan must address the following list, and answering these points will serve as the concrete objectives and takeaways:

\begin{itemize}
\item
  Write down the core problem
\item
  Constrain yourself
  \begin{itemize}
  \item
    one idea in a number of different categories
  \item
    look for ideas in extraneous or expensive solutions - try to think outside of the box
  \end{itemize}
\item
  Aim for 20 ideas
\item
  Take a break - split it up across two 30 minutes brainstorming sessions
\item
  Divide and conquer - consider not only global nav, but sub nav and alternatives to nav as well
\end{itemize}


\section{Brainstorming Execution}
\paragraph{The core problem} navigation/screen real-estate optimization on a mobile browser.

\begin{enumerate}
\item
  \textbf{Focus on mobile browsers} - I want to focus on mobile browsers first since nearly all survey participants (from needfinding in Assignment M2) said they would access the site this way.
\item
  \textbf{Map mobile to desktop} - focus on mobile first and map that design to a desktop interface. Starting with this intention will cause the UI to have screen real estate optimization at it's core.
\item
  \textbf{Why mobile => desktop} - moving from a constricted interface to a more liberal interface.
\end{enumerate}


\clearpage


\paragraph{Constrain yourself} consider nav in a global sense and within context of single page. Ideas that are intended to be outside the box/extraneous/expensive will be accompanied with an asterisk (*). This will be good to keep track that I am staying true to this plan.

\paragraph{Global Navigation} this will be fixed (always available) from everywhere in the app. I will not deviate from this as an attempt to constrain the brainstorming. It is important that a user feel free to navigate to any part of the app at all times. During needfinding, this was one issue with the Groupon interface in that the user felt trapped at times in a subpage since the nav changed.
\begin{enumerate}
\item
  \textbf{Avoid hamburger} - been there done that. Every mobile application has implemented the hamburger condensed nav in the upper right/left corner. This is probably a default responsive feature to popular style frameworks like Bootstrap. The point of this brainstorming session is to rethink nav and come up with a better solution to the hamburger.
\item
  \textbf{Add a search bar}
  \begin{enumerate}
  \item
    \textbf{Fixed} - a search bar that is fixed at all times (under all pages)
  \item
    \textbf{Dynamic size} - represent search bar with icon and have it expand when on focus/hover (hover only for desktop)
  \item
    \textbf{Fixed size} - search bar is fixed
  \item
    \textbf{Suggested searches} - when focused, suggest searches/pages in app
  \item
    \textbf{Metrics driven suggestions} - use metrics from client sessions to track most popular searches and pages and fuel suggested searches from these results
  \item
    \textbf{Previous activity suggestions} - fuel suggested searches from previous activity (store previous searches in localStorage for use later on revisit - caveat, this will not work for people using private browsing)
  \end{enumerate}
\item
  \textbf{Add fixed shorted nav}
  \begin{enumerate}
  \item
    \textbf{Bottom / top} - maybe A/B test which is better - putting the nav on top or on bottom of screen
  \item
    \textbf{Side-to-side} - maybe A/B test (A/B/C/D) with left/right nav
  \item
    \textbf{Left vs right} - right nav will be easy for right handers (majority of population). Will having it on your thumb side cause accidental clicking resulting in sporadic navigating?
  \item
    \textbf{What does best mean?} How do we determine (from A/B testing metrics) what best is? Most use within site? Least use (past more than one nav)?
  \item
    \textbf{Combo with search} - add search in-line with nav
  \item
    \textbf{Dynamic nav} - fuel nav selections from metrics/previous page views (localStorage)
  \item
    \textbf{Nav size} - A/B test using 3 or 4 options in nav
  \item
    \textbf{Screen real-estate} - Consider screen real-estate with all nav location options. Elderly users need proper spacing due to not having a steady hand
  \end{enumerate}
\end{enumerate}

\paragraph{Intra-page Navigation} navigate on current page. Solve the problem of making content on a single page as accessible as possible with real-estate left over from global nav/search

\begin{enumerate}
\item
  \textbf{Profiles}
  \begin{enumerate}
  \item
    \textbf{Tiles that expand} - using a grid, display subpages as tiles - ensure they appear clickable
  \item
    \textbf{Banners that expand} - using banners, display subpages that expand downward - ensure they appear clickable
  \item
    \textbf{Avoid modals} - in lieu of modals leverage animation that expands the tile/banner
  \end{enumerate}
  \textbf{Services / Prices / Products}
  \begin{enumerate}
  \item
    \textbf{Binary?} If only two options, maybe a binary representation would be more interesting?
  \item
    \textbf{Picture correlation} - add pictures from past clients that correspond to services
  \end{enumerate}
  \textbf{Contacts / Hours / Location}
  \begin{enumerate}
  \item
    \textbf{Stay consistent} - however I decide to divide these sub pages - do it the same here. The user should feel a consistent design - they will adapt quicker.
  \item
    \textbf{Stay consistent}
  \end{enumerate}
\end{enumerate}


\section{Selection Criteria}
After brainstorming several alternatives, detail the selection criteria you will use to select which three ideas to move forward to prototyping. This may take the form of the rules that will be applied to selecting the alternatives to move forward, or this may take the form of an explanation of the more situated reasoning behind why certain alternatives are selected. In short, explain how the alternatives to move to prototyping either will be or were selected.
During this phase, specifically connect your selection criteria to the requirements definition or data inventory you posed in Assignment M2. For example, if one of your requirements was that the interface be affordable, then alternatives related to expensive hardware might be avoided.


\section{Prototype}

\begin{enumerate}
\item
  A textual prototype: a plaintext description of the idea, how it will work, what its functionality will be, etc. A textual prototype should be sufficiently detailed to get feedback. This is well-suited for many types of design alternatives.
\item
  A verbal prototype: although this would be presented in text, a verbal prototype would take the form of a loose conversation script about the questions you might ask a person, the answers you would anticipate, the branches you would plan, etc. A verbal prototype is intended to be more dynamic and interactive than a textual prototype. This is well-suited for many types of design alternatives.
\item
  A paper prototype or wireframe: a hand-drawn or simplistic wireframe of the interface you intend to create. It should be thorough enough to get user feedback on its design, but not so detailed that revision would require significant effort; after all, the goal is to get feedback. This is particularly well-suited for a desktop program, tablet app, or web site.
\item
  A card prototype: a collection of smaller screens that can be iteratively shown to a user to simulate their interaction. For example, a user might be asked what they would do when faced with one screen, and then shown a different card representing the screen that would result. This is particularly well-suited for mobile apps where there is less screen real estate.
\item
  A Wizard of Oz prototype: a script of the instructions you would give a user, the commands or actions you would anticipate them making, and the responses you would return to those corresponding actions. This is particularly well-suited to prototypes that do not involve a visible screen, like auditory interfaces.
\item
  A physical prototype: a physical object that allows you to mimic the actions a user might take with your prototype. Rather than having the prototype actually operate as intended, you would note the ease with which the user interacts with the physical device, and potentially simulate the results of different actions. This is particularly well-suited to prototypes for new hardware.
\end{enumerate}


\subsection{Prototype 1}
Note that for most of these types (text, verbal, Wizard of Oz, physical), we would expect a textual description of the prototype in the body of the paper. For a paper prototype or wireframe, we would generally expect one page of your assignment to be the prototype, with around a half-page description/explanation of the prototype following. For a card prototype, we would similarly expect a portion of the prototype in the paper, but you may need to place the rest of the prototype in an appendix.
After creating the prototype, evaluate it from the perspective of the requirements you gathered in Assignment M2. Which requirements does it meet? Which requirements does it miss? How well does the prototype mesh with the audience described in your data inventory?


\subsection{Prototype 2}
Note that for most of these types (text, verbal, Wizard of Oz, physical), we would expect a textual description of the prototype in the body of the paper. For a paper prototype or wireframe, we would generally expect one page of your assignment to be the prototype, with around a half-page description/explanation of the prototype following. For a card prototype, we would similarly expect a portion of the prototype in the paper, but you may need to place the rest of the prototype in an appendix.
After creating the prototype, evaluate it from the perspective of the requirements you gathered in Assignment M2. Which requirements does it meet? Which requirements does it miss? How well does the prototype mesh with the audience described in your data inventory?


\subsection{Prototype 3}
Note that for most of these types (text, verbal, Wizard of Oz, physical), we would expect a textual description of the prototype in the body of the paper. For a paper prototype or wireframe, we would generally expect one page of your assignment to be the prototype, with around a half-page description/explanation of the prototype following. For a card prototype, we would similarly expect a portion of the prototype in the paper, but you may need to place the rest of the prototype in an appendix.
After creating the prototype, evaluate it from the perspective of the requirements you gathered in Assignment M2. Which requirements does it meet? Which requirements does it miss? How well does the prototype mesh with the audience described in your data inventory?

\end{document}

\input{ main-style }

\title{Assignment M3\\}

\authoremail{Jeremy Martinez}{jmartinez91@gatech.edu}

\begin{document}
\maketitle
\thispagestyle{fancy}

\begin{abstract}
I will be focusing on a website’s navigation bar. The nav bar is usually the main call to action when visiting a homepage of a web application and provides context for what actions are available. I will compare navigability between desktop/mobile versions of a website as well.
\end{abstract}


\section{Brainstorming Plan}
The brainstorming plan must address the following list, and answering these points will serve as the concrete objectives and takeaways:

\begin{itemize}
\item
  Write down the core problem
\item
  Constrain yourself
  \begin{itemize}
  \item
    one idea in a number of different categories
  \item
    look for ideas in extraneous or expensive solutions - try to think outside of the box
  \end{itemize}
\item
  Aim for 20 ideas
\item
  Take a break - split it up across two 30 minutes brainstorming sessions
\item
  Divide and conquer - consider not only global nav, but sub nav and alternatives to nav as well
\end{itemize}


\section{Brainstorming Execution}
\paragraph{The core problem} navigation/screen real estate optimization on a mobile browser.

\begin{enumerate}
\item
  \textbf{Focus on mobile browsers} - I want to focus on mobile browsers first since nearly all survey participants (from needfinding in Assignment M2) said they would access the site this way.
\item
  \textbf{Map mobile to desktop} - focus on mobile first and map that design to a desktop interface. Starting with this intention will cause the UI to have screen real estate optimization at it's core.
\item
  \textbf{Why mobile => desktop} - moving from a constricted interface to a more liberal interface.
\end{enumerate}


\clearpage


\paragraph{Constrain yourself} consider nav in a global sense and within context of single page. Ideas that are intended to be outside the box/extraneous/expensive will be accompanied with an asterisk (*). This will be good to keep track that I am staying true to this plan.

\paragraph{Global Navigation} this will be fixed (always available) from everywhere in the app. I will not deviate from this as an attempt to constrain the brainstorming. It is important that a user feel free to navigate to any part of the app at all times. During needfinding, this was one issue with the Groupon interface in that the user felt trapped at times in a subpage since the nav changed.
\begin{enumerate}
\item
  \textbf{Avoid hamburger} - been there done that. Every mobile application has implemented the hamburger condensed nav in the upper right/left corner. This is probably a default responsive feature to popular style frameworks like Bootstrap. The point of this brainstorming session is to rethink nav and come up with a better solution to the hamburger.
\item
  \textbf{Add a search bar}
  \begin{enumerate}
  \item
    \textbf{Fixed} - a search bar that is fixed at all times (under all pages)
  \item
    \textbf{Dynamic size} - represent search bar with icon and have it expand when on focus/hover (hover only for desktop)
  \item
    \textbf{Fixed size} - search bar is fixed
  \item
    \textbf{Suggested searches} - when focused, suggest searches/pages in app
  \item
    \textbf{Metrics driven suggestions} - use metrics from client sessions to track most popular searches and pages and fuel suggested searches from these results
  \item
    \textbf{Previous activity suggestions} - fuel suggested searches from previous activity (store previous searches in localStorage for use later on revisit - caveat, this will not work for people using private browsing)
  \end{enumerate}
\item
  \textbf{Add fixed shorted nav}
  \begin{enumerate}
  \item
    \textbf{Bottom / top} - maybe A/B test which is better - putting the nav on top or on bottom of screen
  \item
    \textbf{Side-to-side} - maybe A/B test (A/B/C/D) with left/right nav
  \item
    \textbf{Left vs right} - right nav will be easy for right handers (majority of population). Will having it on your thumb side cause accidental clicking resulting in sporadic navigating?
  \item
    \textbf{What does best mean?} How do we determine (from A/B testing metrics) what best is? Most use within site? Least use (past more than one nav)?
  \item
    \textbf{Combo with search} - add search in-line with nav
  \item
    \textbf{Dynamic nav} - fuel nav selections from metrics/previous page views (localStorage)
  \item
    \textbf{Nav size} - A/B test using 3 or 4 options in nav
  \item
    \textbf{Screen real estate} - Consider screen real estate with all nav location options. Elderly users need proper spacing due to not having a steady hand
  \end{enumerate}
\end{enumerate}

\paragraph{Intra-page Navigation} navigate on current page. Solve the problem of making content on a single page as accessible as possible with real estate left over from global nav/search

\begin{enumerate}
\item
  \textbf{Profiles}
  \begin{enumerate}
  \item
    \textbf{Tiles that expand} - using a grid, display subpages as tiles - ensure they appear clickable
  \item
    \textbf{Banners that expand} - using banners, display subpages that expand downward - ensure they appear clickable
  \item
    \textbf{Avoid modals} - in lieu of modals leverage animation that expands the tile/banner
  \end{enumerate}
  \textbf{Services / Prices / Products}
  \begin{enumerate}
  \item
    \textbf{Binary?} If only two options, maybe a binary representation would be more interesting?
  \item
    \textbf{Picture correlation} - add pictures from past clients that correspond to services
  \end{enumerate}
  \textbf{Contacts / Hours / Location}
  \begin{enumerate}
  \item
    \textbf{Stay consistent} - however I decide to divide these subpages - do it the same here. The user should feel a consistent design - they will adapt quicker.
  \item
    \textbf{Stay consistent}
  \end{enumerate}
\end{enumerate}


\section{Selection Criteria}
Moving from brainstorming to prototyping, I want to sketch out a basis for the interface and then build upon that. These three starting points will have the "how" functionality of the nav as its main concern. From brainstorming, the main takeaways that I want to focus on when exploring prototypes will be

\begin{enumerate}
\item
  \textbf{Maximize screen real estate} functionality without too much clutter AND \textbf{Avoid modals} - these are jarring
\item
  \textbf{Pick one main feature} of the nav and build around that
  \begin{enumerate}
  \item
    \textbf{Fixed nav at bottom} - I discussed during brainstorming where this nav could live and I think the bottom makes the most sense. The reason for this is that the thumb is already located there (not the top). And the screen is tall and narrow, so conserving space on the sides is important.
  \item
    \textbf{Search bar} - all prototypes MUST have a search. I will toggle between an expanding search (from an icon) and a fixed search bar at the top (maybe something that can be A/B tested and swapped in/out with other designs).
  \item
    \textbf{Central nav page} - this is something I did not address in brainstorming, and may feel like a bit more of an extreme implementation. Having an isolated page for the nav will widen the gulf of execution. However, if implemented properly, maybe this can look and feel a bit like a traditional top nav bar.
  \item
    \textbf{Carousel} - this is another rather different approach to a web app. This borrows the Apple iPhone homepage implementation by paginating in a horizonal fashion.
  \end{enumerate}
\end{enumerate}

These are the concrete takeaways from brainstorming that I will use as a starting point for prototyping. However, I will continue to brainstorm as I am implementing these \textemph{rough} prototypes and use more of the findings to the designs.

\paragraph{Data inventory} tying this back in to our users (see appendix A), the prototypes here consider the following

\begin{enumerate}
\item
  Focusing on mobile first targets our age range of whom will default to a mobile browser over a desktop/laptop browser.
\item
  The Carousel prototype specifically will leverage their expertise in smart phone UI familiarity. Since this will match the look and feel of their mobile phone's operating system.
\item
  Information gathering via page visit metrics and leveraging prior user behavior via local storage in the browser will target their motivation for engaging in the website. This will attempt to start the user closer to their intended goal to narrow the gulf of execution.
\item
  Metrics around the search results will help us identify any gaps in what the user may want/need that the interface is not providing/making clear.
\item
  Addressing the "where the users are" is tricky, since I want to avoid the browser from asking for permission to their location. This can feel invasive to the user, especially if they do not realize why we are asking for this information.
\end{enumerate}

\section{Prototypes}

For all prototypes, the numbers are an abstract representation to be replaced by either an icon or a icon/label pair. Some example icons and labels can be seen in Figure 1 for the first prototype. These will be reference throughout all three prototypes.

\subsection{Prototype 1}
\paragraph{Description} The first prototype is a paper drawn wireframe of four example screens in the mobile interface. They are sequenced, moving from the upper left drawing, to lower right, to upper right, to lower left.

The nav is fixed and on the bottom of the screen, displaying one of two lists. If the user is navigating to this page for the first time, we will fill these three options with the three most frequently visited pages of the application. As the user clicks and navigates to other pages, we will record this in their own browser's local storage. If the user is a repeat visitor, then we will read this local storage and populate the nav based off of what pages they visited (most frequently if more than three) on their last visit.

The search bar will be fixed as well at the top. However, it's display will be dynamic depending on whether it is in focus or not. Clicking the question mark will cause the search bar to slide out from the icon to the left. Then, the user can search the entire website's content from here.

\paragraph{Evaluation} This addresses the requirements discovered during needfinding in assignment M2 because it leverages the access to the nav the users liked about YouTube's mobile app. It also spaces everything out for our users making it more difficult to accidentally click something they are not intending to. This meshes well with the data inventory in that it addresses all bullet points from the inventory list in a minimal gulf of execution.

\begin{figure}[H]
  \centering
  \includegraphics[scale=0.07]{figs/prototype-1}
  \caption{Prototype 1 - Fixed nav with search icon}
  \label{fig::1}
\end{figure}


\subsection{Prototype 2}
\paragraph{Description} The second prototype is a paper drawn wireframe of five example screens in the mobile interface. They are sequenced, moving from the upper left to the right, top to bottom.

The nav is actually not fixed on the page but placed at the top of every page in a grid pattern. The top 6 most common choices are displayed here. Selecting an option will render the relevant content below the grid, as well as auto-scroll the user down below the grid to the where the content is (this action is represented from screen 1 -> 2).

As the user scrolls through the pages content, a call to action to scroll back to the grid (top of the page) will remain fixed at the top of the screen (this is clicked in screen 3 of the prototype). A search bar will remain at the top of the grid (top of the page) at all times. Then, the user can search the entire website's content from here. Icons on the grid may change according to results from search bar.

\paragraph{Evaluation} This addresses the requirements of always having a nav available, however, it does so in a slightly different way. The auto scrolling will need to flow smoothly in order to no be too jarring and appear intuitive to the user where they are at all times. It also satisfies the requirement of having a dynamic search at all times. It does lengthen a simple task of navigating to a page (lengthening the gulf of execution). However, an intuitive design and animation may make up for this gulf by increasing intuition.

This matches our data inventory in that it surfaces quickly the user's main motivation for being there. The calls to action, like in the first prototype, are driven by metrics. This allows us to dynamically read our users goals and displays the tasks/subtasks necessary to accomplish them. The scrolling helps the user detect where they are relative to components on the screen.

\begin{figure}[H]
  \centering
  \includegraphics[scale=0.07]{figs/prototype-2}
  \caption{Prototype 2 - Grid with search bar}
  \label{fig::1}
\end{figure}


\subsection{Prototype 3}
\paragraph{Description} The third prototype is a paper drawn wireframe of five example screens in the mobile interface. They are sequenced, moving from the upper left to the right, top to bottom.

This nav is fixed at the bottom of the page. The top 5-7 choices (these numbers are arbitrary and user tests will help us get an exact value) are displayed in the nav. These are represented as dots at the bottom of the screen. The user can scroll vertically for the individual pages content. They can navigate pages by swiping horizontally. The dots match a UI similar to the iPhone home screen. The user can also click on the nav. Doing so will magnify the options and they can scroll horizontally through the options to find the page they are looking for. A search bar will remain at the top of the grid (top of the page) at all times. Then, the user can search the entire website's content from here. Icons on the grid may change according to results from search bar.

Of all the nav bars, this feels like the one with the least friction. Comparatively, it takes a minimalist approach, and therefore, may lack in intuition. Another concern may be with the gulf of evaluation. Since this design differs so drastically from other mobile apps, the reaction in the UI to every action by the user must be obvious and reflect the importance of the action.

\paragraph{Evaluation} This addresses the requirements of always having a nav available. The auto scrolling potentially introduces a larger gulf of execution if the user starts on page 1 and wants to get to page 7. They will have to either scroll all pages, or click the nav, scroll the nav, and click the seventh icon. Even though this gulf is larger, it is consistent with most mobile operating systems and offers (out of all prototypes) the most screen real estate for content. One addition is we can save in the browser state what page they last visited. If they were on page 6 the last time they used the app, then when they return, they will start at page 6.

This matches our data inventory in that it addresses their motivation for being there. Their goals are to perform some task or obtain information on one of the pages. Since these pages are so readily available, their goal is addressed in the interface in a concise manner.

\begin{figure}[H]
  \centering
  \includegraphics[scale=0.07]{figs/prototype-3}
  \caption{Prototype 3 - Carousel with search bar}
  \label{fig::1}
\end{figure}





\clearpage

\section{Appendix A - Data inventory}

\paragraph{Our expectation on user types}
The intended user for this application is adults in general. However, the current customer base will be the heavy focus of whom this experience will be designed for, which includes men and women ages 30-60. We anticipate the majority of the target audience to be older than millennials. When evaluating the intended audience, the following is taken into account

\begin{itemize}
\item
  \textbf{Demographic of our audience} fits (for the most part) men and women aged 25-65. We are choosing this because, for the specific salon owner I am working with, their client-base has matched this for the last 20 years.
\item
  \textbf{Their level of expertise} will be assumed to be not very technical. We will assume for this reason that the user does not have a strong grasp on the differences between types of browsers, versions of browsers, or experiences between mobile vs desktop browsers. The UI should be styled in such a way that the look and feel is familiar, yet clearly geared to be optimized on the platform at hand.
\item
  \textbf{Their motivation for engaging in the task} will not be singular. The number of reasons one might visit a hair salons website could be for general salon information, hours of operation, salon team members, products/services they offer, schedule an appointment, contact information. We can try to anticipate our user's needs; however, we must assume they have any one of these intentions. Therefore, having an intuitive navigation inside the application is crucial to naturally enabling the user to accomplish the task.
\end{itemize}

\paragraph{Who are the users?}
Our users are new and existing customers. Depending on which, they may have slightly different goals, however, this is not a drastic difference. Our users are men and women between the ages of 25-65.

\paragraph{Where are the users?}
They are on their phones! A small percentage will be on their laptops/desktops; however, this is a clear minority.

\paragraph{What is the context of the task?}
The context of the task is landing on a business home webpage (usually from Google/Yelp/some other search engine like service).

\paragraph{What are their goals?}
They typically have one goal - to find a new hair salon or to get information on their existing hair salon.

\paragraph{What do they need?}
They need a responsive app (for both mobile/tablet/desktop). They need clear calls to action on their landing page to quickly accomplish their task. They need more icons/images and less font. They need a nav bar on both top and bottom, so these CTA are always readily available.

\paragraph{What are their tasks?}
Their tasks are to find a salon portfolio, types of products/services, bio of stylists, prices, online booking, free consultation, blogs, hours, contact info, business history, etc.

\paragraph{What are their subtasks?}
Subtasks for these embody the steps in the gulf of execution. With the right design and nav bar, these subtasks can be reduced every time to one or two clicks.


\section{Appendix B - Types of Prototypes}

\begin{enumerate}
\item
  A textual prototype: a plaintext description of the idea, how it will work, what its functionality will be, etc. A textual prototype should be sufficiently detailed to get feedback. This is well-suited for many types of design alternatives.
\item
  A verbal prototype: although this would be presented in text, a verbal prototype would take the form of a loose conversation script about the questions you might ask a person, the answers you would anticipate, the branches you would plan, etc. A verbal prototype is intended to be more dynamic and interactive than a textual prototype. This is well-suited for many types of design alternatives.
\item
  A paper prototype or wireframe: a hand-drawn or simplistic wireframe of the interface you intend to create. It should be thorough enough to get user feedback on its design, but not so detailed that revision would require significant effort; after all, the goal is to get feedback. This is particularly well-suited for a desktop program, tablet app, or web site.
\item
  A card prototype: a collection of smaller screens that can be iteratively shown to a user to simulate their interaction. For example, a user might be asked what they would do when faced with one screen, and then shown a different card representing the screen that would result. This is particularly well-suited for mobile apps where there is less screen real estate.
\item
  A Wizard of Oz prototype: a script of the instructions you would give a user, the commands or actions you would anticipate them making, and the responses you would return to those corresponding actions. This is particularly well-suited to prototypes that do not involve a visible screen, like auditory interfaces.
\item
  A physical prototype: a physical object that allows you to mimic the actions a user might take with your prototype. Rather than having the prototype actually operate as intended, you would note the ease with which the user interacts with the physical device, and potentially simulate the results of different actions. This is particularly well-suited to prototypes for new hardware.
\end{enumerate}

\end{document}
